\documentclass[]{article}
\usepackage{lmodern}
\usepackage{amssymb,amsmath}
\usepackage{ifxetex,ifluatex}
\usepackage{fixltx2e} % provides \textsubscript
\ifnum 0\ifxetex 1\fi\ifluatex 1\fi=0 % if pdftex
  \usepackage[T1]{fontenc}
  \usepackage[utf8]{inputenc}
\else % if luatex or xelatex
  \ifxetex
    \usepackage{mathspec}
  \else
    \usepackage{fontspec}
  \fi
  \defaultfontfeatures{Ligatures=TeX,Scale=MatchLowercase}
\fi
% use upquote if available, for straight quotes in verbatim environments
\IfFileExists{upquote.sty}{\usepackage{upquote}}{}
% use microtype if available
\IfFileExists{microtype.sty}{%
\usepackage{microtype}
\UseMicrotypeSet[protrusion]{basicmath} % disable protrusion for tt fonts
}{}
\usepackage[margin=1in]{geometry}
\usepackage{hyperref}
\hypersetup{unicode=true,
            pdftitle={Proiect},
            pdfborder={0 0 0},
            breaklinks=true}
\urlstyle{same}  % don't use monospace font for urls
\usepackage{graphicx,grffile}
\makeatletter
\def\maxwidth{\ifdim\Gin@nat@width>\linewidth\linewidth\else\Gin@nat@width\fi}
\def\maxheight{\ifdim\Gin@nat@height>\textheight\textheight\else\Gin@nat@height\fi}
\makeatother
% Scale images if necessary, so that they will not overflow the page
% margins by default, and it is still possible to overwrite the defaults
% using explicit options in \includegraphics[width, height, ...]{}
\setkeys{Gin}{width=\maxwidth,height=\maxheight,keepaspectratio}
\IfFileExists{parskip.sty}{%
\usepackage{parskip}
}{% else
\setlength{\parindent}{0pt}
\setlength{\parskip}{6pt plus 2pt minus 1pt}
}
\setlength{\emergencystretch}{3em}  % prevent overfull lines
\providecommand{\tightlist}{%
  \setlength{\itemsep}{0pt}\setlength{\parskip}{0pt}}
\setcounter{secnumdepth}{5}
% Redefines (sub)paragraphs to behave more like sections
\ifx\paragraph\undefined\else
\let\oldparagraph\paragraph
\renewcommand{\paragraph}[1]{\oldparagraph{#1}\mbox{}}
\fi
\ifx\subparagraph\undefined\else
\let\oldsubparagraph\subparagraph
\renewcommand{\subparagraph}[1]{\oldsubparagraph{#1}\mbox{}}
\fi

%%% Use protect on footnotes to avoid problems with footnotes in titles
\let\rmarkdownfootnote\footnote%
\def\footnote{\protect\rmarkdownfootnote}

%%% Change title format to be more compact
\usepackage{titling}

% Create subtitle command for use in maketitle
\newcommand{\subtitle}[1]{
  \posttitle{
    \begin{center}\large#1\end{center}
    }
}

\setlength{\droptitle}{-2em}
  \title{Proiect}
  \pretitle{\vspace{\droptitle}\centering\huge}
  \posttitle{\par}
  \author{}
  \preauthor{}\postauthor{}
  \date{}
  \predate{}\postdate{}

\usepackage{booktabs}
\usepackage{longtable}
\usepackage{framed,color}
\definecolor{shadecolor}{RGB}{248, 248, 248}
\definecolor{shadecolor1}{RGB}{216,225,235}
\definecolor{framecolor}{RGB}{108,123,13}

\ifxetex
  \usepackage{letltxmacro}
  \setlength{\XeTeXLinkMargin}{1pt}
  \LetLtxMacro\SavedIncludeGraphics\includegraphics
  \def\includegraphics#1#{% #1 catches optional stuff (star/opt. arg.)
    \IncludeGraphicsAux{#1}%
  }%
  \newcommand*{\IncludeGraphicsAux}[2]{%
    \XeTeXLinkBox{%
      \SavedIncludeGraphics#1{#2}%
    }%
  }%
\fi

\newenvironment{frshaded*}{%
  \def\FrameCommand{\fboxrule=\FrameRule\fboxsep=\FrameSep \fcolorbox{framecolor}{shadecolor1}}%
  \MakeFramed {\advance\hsize-\width \FrameRestore}}%
{\endMakeFramed}

\newenvironment{rmdblock}[1]
  {\begin{frshaded*}
  \begin{itemize}
  \renewcommand{\labelitemi}{
    \raisebox{-.7\height}[0pt][0pt]{
      {\setkeys{Gin}{width=2em,keepaspectratio}\includegraphics{images/icons/#1}}
    }
  }
  \item
  }
  {
  \end{itemize}
  \end{frshaded*}
  }
  
\newenvironment{rmdcaution}
  {\begin{rmdblock}{caution}}
  {\end{rmdblock}}
% \newenvironment{rmdinsight}
%   {\begin{rmdblock}{insight}}
%   {\end{rmdblock}}
\newenvironment{rmdexercise}
  {\begin{rmdblock}{exercise}}
  {\end{rmdblock}}
\newenvironment{rmdtip}
  {\begin{rmdblock}{tip}}
  {\end{rmdblock}}
  
  
%%%%%%%%%%%%%%%%%%%%%%%%%%%%%%%%%%%%%%%%%%%%%%%%%%%%%%%%%%%%%%%%%%%%%%%%%%%%%%%%%%%%%%%%%%%%%%%%%%%%%%%%%%%%%%%%%%%%%
%%%%%%%%%%% For insight block %%%%%%%%%%%%%%%%%%%%%%%%%%
\definecolor{shadecolor_insight}{RGB}{223,240,216}
\definecolor{framecolor_insight}{RGB}{136,193,137}

\newenvironment{frshaded_insight*}{%
  \def\FrameCommand{\fboxrule=\FrameRule\fboxsep=\FrameSep \fcolorbox{framecolor_insight}{shadecolor_insight}}%
  \MakeFramed {\advance\hsize-\width \FrameRestore}}%
{\endMakeFramed}

\newenvironment{rmdblock_insight}[1]
  {\begin{frshaded_insight*}
  \begin{itemize}
  \renewcommand{\labelitemi}{
    \raisebox{-.7\height}[0pt][0pt]{
      {\setkeys{Gin}{width=2em,keepaspectratio}\includegraphics{images/icons/#1}}
    }
  }
  \item
  }
  {
  \end{itemize}
  \end{frshaded_insight*}
  }
  
\newenvironment{rmdinsight}
  {\begin{rmdblock_insight}{insight}}
  {\end{rmdblock_insight}}
  
%%%%%%%%%%%%%%%%%%%%%%%%%%%%%%%%%%%%%%%%%%%%%%%%%%%%%%%%%%%%%%%%%%%%%%%%%%%%%%%%%%%%%%%%%%%%%%%%%%%%%%%%%%%%%%%%%%%%%
\usepackage{subfigure}
\usepackage{booktabs}
\usepackage{slashbox}
\usepackage{color}
%%%%%%%%%%%%%%%%%%%%%%%%%%%%%%%%%%%%%%%%%%%%%%%%%%%%%%%%%%%%%%%%%%%%%%%%%%%%%%%%%%%%%%%%%%%%%%%%%%%%%%%%%%%%%%%%%%%%%
%CITEVA DEFINITII
\def\om{\omega}
\def\Om{\Omega}
\def\et{\eta}
\def\td{\tilde{\delta}}
\def\m{{\mu}}
\def\n{{\nu}}
\def\k{{\kappa}}
\def\l{{\lambda}}
\def\L{{\Lambda}}
\def\g{{\gamma}}
\def\a{{\alpha}}
\def\e{{\varepsilon}}
\def\b{{\beta}}
\def\G{{\Gamma}}
\def\d{{\delta}}
\def\D{{\Delta}}
\def\t{{\theta}}
\def\s{{\sigma}}
\def\S{{\Sigma}}
\def\z{{\zeta}}
\def\qed{\hfill\Box}
\def\ds{\displaystyle}
\def\mc{\mathcal}
%%%%%%%%%%%%%%%%%%%%%%%%%%%%%%%%%%%%%%%%%%%%%%%%%%%%%%%%%%%%%%%%%%%%%%%%%%%%%%%%%%%%%%%%%%%%%%%%%%%%%%%%%%%%%%%%%%%%%%
\def\1{{\mathbf 1}}
\def\CC{{\mathbb C}}
\def\VV{{\mathbb V}}
\def\RR{{\mathbb R}}
\def\QQ{{\mathbb Q}}
\def\ZZ{{\mathbb Z}}
\def\PP{{\mathbb P}}
\def\EE{{\mathbb E}}
\def\NN{{\mathbb N}}
\def\FF{{\mathbb F}}
%\def\SS{{\mathbb S}}
\def\MA{{\mathcal A}}
\def\MO{{\mathcal O}}
\def\MF{{\mathcal F}}
\def\ME{{\mathcal E}}
\def\MR{{\mathcal R}}
\def\MB{{\mathcal B}}
\def\MM{{\mathcal M}}
\def\MN{{\mathcal N}}
\def\MU{{\mathcal U}}
\def\MP{{\mathcal P}}
\def\MS{{\mathcal S}}
\def\MBS{{\mathbf S}}
\def\MX{{\bm{ \mathscr X}}}

% independent sign
\newcommand\independent{\protect\mathpalette{\protect\independenT}{\perp}}
\def\independenT#1#2{\mathrel{\rlap{$#1#2$}\mkern2mu{#1#2}}}

%%%%%%%%%%%%%%%%%%%%%%%%%%%%%%%%%%%%%%%%%%%%%%%%%%%%%%%%%%%%%%%%%%%%%%%%%%%%%%%%%%%%%%%%%%%%%%%%%%%%%%%%%%%%%%%%%%%%%
%Header and Footer
\usepackage{fancyhdr}

\pagestyle{fancy}
\fancyhf{}
\rhead{Universitatea din Bucure\c sti\\ Facultatea de Matematic\u a \c si Informatic\u a}
\lhead{\textit{Curs}: Probabilit\u a\c ti \c si Statistic\u a\\ \textit{Instructor}: A. Am\u arioarei}
\rfoot{Pagina \thepage}
\lfoot{Grupele: 241, 242, 243, 244}
%%%%%%%%%%%%%%%%%%%%%%%%%%%%%%%%%%%%%%%

\usepackage{movie15}
\usepackage{booktabs}
\usepackage{longtable}
\usepackage{array}
\usepackage{multirow}
\usepackage[table]{xcolor}
\usepackage{wrapfig}
\usepackage{float}
\usepackage{colortbl}
\usepackage{pdflscape}
\usepackage{tabu}
\usepackage{threeparttable}

\begin{document}
\maketitle

%%%%%%%%%%%%%%%%%%%%%%%%
\thispagestyle{fancy}

\textbf{Notă:} Raportul poate fi scris în \emph{Microsoft Word} sau
\LaTeX (pentru ușurință recomand folosirea pachetului \emph{rmarkdown}
din \emph{R} - mai multe informații găsiți pe site la secțiunea
\emph{Link-uri utile}). Toate simulările, figurile și codurile folosite
trebuie incluse în raport. Se va folosi doar limbajul \emph{R}.

\section{Problema 1}\label{problema-1}

Considerăm următoarele distribuții: \(\operatorname{Bin}(n,p)\),
\(\operatorname{Pois}(\lambda)\), \(\operatorname{Exp}(\lambda)\),
\(\mathcal{N}(\mu, \sigma^2)\)

\begin{enumerate}
\def\labelenumi{\arabic{enumi}.}
\item
  Generați \(N=1000\) de realizări independente din fiecare repartiție
  și calculați media și varianța eșantionului.
\item
  Ilustrați grafic funcțiile de masă, respectiv funcțiile de densitate
  pentru fiecare din repartițiile din enunțul problemei. Considerați
  câte \(5\) seturi de parametrii diferiți pentru fiecare repartiție și
  suprapuneți graficele pe aceeași figură pentru fiecare rapetiție.
  Adăugați și legenda.
\item
  Pentru seturile de parametrii de la punctul anterior trasați funcțiile
  de repartiție pentru fiecare repartiție (tot suprapuse) și adăugați
  legenda corespunzătoare.
\end{enumerate}

Scopul următoarelor subpuncte este de a evalua acuratețea unor
aproximări ale funcției de repartiție a binomialei \(\mathcal{B}(n,p)\).
Vom compara următoarele patru aproximări (cu excepția aproximării
Camp-Paulson, celelalte trei au fost văzute la curs):

\begin{enumerate}
\def\labelenumi{\alph{enumi})}
\item
  \emph{Aproximarea Poisson}

  \[
    F_{n,p}(k) \approx F_{\lambda}(k)=\sum_{x=0}^{k}e^{-\lambda}\frac{\lambda^x}{x!}, \quad \lambda = np
  \]
\item
  \emph{Aproximarea Normală} (rezultată din Teorema Limită Centrală)

  \[
    F_{n,p}(k) \approx \Phi\left(\frac{k-np}{\sqrt{np(1-p)}}\right)
  \]
\item
  \emph{Aproximarea Normală cu factor de corecție}

  \[
    F_{n,p}(k) \approx \Phi\left(\frac{k+0.5-np}{\sqrt{np(1-p)}}\right)
  \]
\item
  \emph{Aproximarea Camp-Paulson}\footnote{A se vedea articolul lui
    Camp, B. H. - \emph{Approximation to the Point Binomial}, Annals of
    Mathematical Statistics, 22, pp.~130-131, 1951.}

  \[
    F_{n,p}(k) \approx \Phi\left(\frac{c-\mu}{\sigma}\right)
  \]
\end{enumerate}

pentru \(c = (1-b)r^{\frac{1}{3}}\), \(\mu = 1-a\) și
\(\sigma^2 = a+br^{\frac{2}{3}}\) unde \(a = \frac{1}{9(n-k)}\),
\(b = \frac{1}{9(k+1)}\) și respectiv
\(r = \frac{[(k+1)(1-p)]}{[p(n-k)]}\).

Se cere:

\begin{enumerate}
\def\labelenumi{\arabic{enumi}.}
\setcounter{enumi}{3}
\item
  Pentru fiecare \(n\in\{25, 50, 100\}\) și fiecare
  \(p\in\{0.05, 0.1\}\) să se afișeze un tabel cu șase coloane
  (\texttt{k}, \texttt{Binomiala}, \texttt{Poisson}, \texttt{Normala},
  \texttt{Normala\ Corecție}, \texttt{Camp-Paulson}) în care să apară
  aproximările de mai sus pentru funcția de repartiție și de masă a
  binomialei, pentru \(k\in\{1,2,\ldots, 10\}\).
\item
  Pentru a cuantifica acuratețea aproximărilor de mai sus vom folosi ca
  metrică, \emph{eroarea maximală absolută} dintre două funcții de
  repartiții \(F\) și \(H\) (numită și distanța Kolmogorov) dată de
  formula
\end{enumerate}

\[
    d_K(F(k), H(k)) = \max_{k}\left|F(k) - H(k)\right|.
  \] Pentru fiecare \(n\in\{25, 50, 100\}\) ilustrați pe același grafic
erorile maximale absolute (folosind diferite culori și simboluri pentru
puncte) dintre funcția de repartiție binomială și cele patru aproximări
de mai sus considerând \(0.01\leq p\leq 0.5\). Ce observați ?

\section{Problema 2}\label{problema-2}

Obiectivul acestui exercițiu este de a simula un vector aleator
\((X_1,X_2)\) repartizat uniform pe discul unitate \(D(1)\) (discul de
centru \((0,0)\) și de rază \(1\)). Densitatea acestuia este

\[
f_{(X_1,X_2)}(x_1,x_2) = \frac{1}{\pi}\mathbf{1}_{D(1)}(x_1,x_2).
\] Pentru aceasta vom folosi două metode. O primă metodă este metoda de
simulare prin acceptare și respingere. Această metodă este des utilizată
pentru generarea unei v.a. repartizate uniform pe o mulțime oarecare
\(E\). Metoda constă în generarea unei v.a. \(X\) repartizată uniform pe
o mulțime \(F\supset E\) mai simplă decât \(E\), apoi de a testa dacă
\(X\) se află în \(E\) sau nu. În caz afirmativ, păstrăm \(X\) altfel
generăm o nouă realizare a lui \(X\) pe \(F\).

\begin{enumerate}
\def\labelenumi{\arabic{enumi}.}
\item
  Justificați teoretic că putem simula un vector (cuplu) aleator
  repartizat uniform pe pătratul \([-1,1]^1\) plecând de la două v.a.
  independente repartizate uniform pe segmentul \([-1,1]\).
\item
  Prin metoda acceptării și respingerii simulați \(N=1000\) de puncte
  independente repartizate uniform pe discul unitate \(D(1)\).
  Reprezentați grafic punctele \((X_i,Y_i)\) din interiorul discului
  unitate cu albastru, și pe celelalte cu roșu.
\item
  Calculați media aritmetică a distanței care separă cele \(N\) puncte
  de origine. Comparați rezultatul cu media teoritică a variabilei
  corespunzătoare.
\end{enumerate}

O a doua metodă de simulare a unui punct \((X,Y)\) repartizat uniform pe
\(D(1)\) constă în folosirea schimbării de variabilă în coordonate
polare: \(X=R\cos(\Theta)\) și \(Y=R\sin(\Theta)\).

\begin{enumerate}
\def\labelenumi{\arabic{enumi}.}
\setcounter{enumi}{3}
\item
  Plecând de la densitatea cuplului \((X,Y)\), găsiți densitatea v.a.
  \(R\) și \(\Theta\).\footnote{\emph{Indicație:} Aici puteți folosi
    următorul rezultat bazat pe formula de schimbare de variabilă în
    cazul multidimensional: Fie \(\textbf{X}=(X_1, \ldots, X_n)\) un
    vector aleator cu densitatea \(f_{\textbf{X}}\) și
    \(g:\mathbb{R}^n\to\mathbb{R}^n\) o funcție diferențiabilă de clasă
    \(\mathcal{C}^1\), injectivă și cu Jacobianul nenul. Atunci vectorul
    aleator \(\textbf{Y}=g(\textbf{X})\) are densitatea
    \(f_{\textbf{Y}}(y) = f_{\textbf{X}}\left(g^{-1}(y)\right)\left|\det J_{g^{-1}}(y)\right|\)
    dacă \(y\in Im(g)\) și \(0\) altfel}
\item
  Simulați \(N=1000\) de puncte prin această metodă și ilustrați grafic
  aceste puncte (incluzand conturul cercului).
\end{enumerate}


\end{document}
