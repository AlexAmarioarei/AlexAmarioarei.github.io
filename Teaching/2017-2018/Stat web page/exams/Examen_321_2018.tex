\documentclass[]{article}
\usepackage{lmodern}
\usepackage{amssymb,amsmath}
\usepackage{ifxetex,ifluatex}
\usepackage{fixltx2e} % provides \textsubscript
\ifnum 0\ifxetex 1\fi\ifluatex 1\fi=0 % if pdftex
  \usepackage[T1]{fontenc}
  \usepackage[utf8]{inputenc}
\else % if luatex or xelatex
  \ifxetex
    \usepackage{mathspec}
  \else
    \usepackage{fontspec}
  \fi
  \defaultfontfeatures{Ligatures=TeX,Scale=MatchLowercase}
\fi
% use upquote if available, for straight quotes in verbatim environments
\IfFileExists{upquote.sty}{\usepackage{upquote}}{}
% use microtype if available
\IfFileExists{microtype.sty}{%
\usepackage{microtype}
\UseMicrotypeSet[protrusion]{basicmath} % disable protrusion for tt fonts
}{}
\usepackage[margin=1in]{geometry}
\usepackage{hyperref}
\hypersetup{unicode=true,
            pdfborder={0 0 0},
            breaklinks=true}
\urlstyle{same}  % don't use monospace font for urls
\usepackage{graphicx,grffile}
\makeatletter
\def\maxwidth{\ifdim\Gin@nat@width>\linewidth\linewidth\else\Gin@nat@width\fi}
\def\maxheight{\ifdim\Gin@nat@height>\textheight\textheight\else\Gin@nat@height\fi}
\makeatother
% Scale images if necessary, so that they will not overflow the page
% margins by default, and it is still possible to overwrite the defaults
% using explicit options in \includegraphics[width, height, ...]{}
\setkeys{Gin}{width=\maxwidth,height=\maxheight,keepaspectratio}
\IfFileExists{parskip.sty}{%
\usepackage{parskip}
}{% else
\setlength{\parindent}{0pt}
\setlength{\parskip}{6pt plus 2pt minus 1pt}
}
\setlength{\emergencystretch}{3em}  % prevent overfull lines
\providecommand{\tightlist}{%
  \setlength{\itemsep}{0pt}\setlength{\parskip}{0pt}}
\setcounter{secnumdepth}{0}
% Redefines (sub)paragraphs to behave more like sections
\ifx\paragraph\undefined\else
\let\oldparagraph\paragraph
\renewcommand{\paragraph}[1]{\oldparagraph{#1}\mbox{}}
\fi
\ifx\subparagraph\undefined\else
\let\oldsubparagraph\subparagraph
\renewcommand{\subparagraph}[1]{\oldsubparagraph{#1}\mbox{}}
\fi

%%% Use protect on footnotes to avoid problems with footnotes in titles
\let\rmarkdownfootnote\footnote%
\def\footnote{\protect\rmarkdownfootnote}

%%% Change title format to be more compact
\usepackage{titling}

% Create subtitle command for use in maketitle
\newcommand{\subtitle}[1]{
  \posttitle{
    \begin{center}\large#1\end{center}
    }
}

\setlength{\droptitle}{-2em}
  \title{\textbf{Examen}}
  \pretitle{\vspace{\droptitle}\centering\huge}
  \posttitle{\par}
  \author{}
  \preauthor{}\postauthor{}
  \predate{\centering\large\emph}
  \postdate{\par}
  \date{11 Februarie 2018}

%%%%%%%%%%%%%%%%%%%%%%%%%%%%%%%%%%%%%%%%%%%%%%%%%%%%%%%%%%%%%%%%%%%%%%%%%%%%%%%%%%%%%%%%%%%%%%%%%%%%%%%%%%%%%%%%%%%%%
\usepackage{subfigure}
\usepackage{booktabs}
\usepackage{slashbox}
\usepackage{color}

%%%%%%%%%%%%%%%%%%%%%%%%%%%%%%%%%%%%%%%%%%%%%%%%%%%%%%%%%%%%%%%%%%%%%%%%%%%%%%%%%%%%%%%%%%%%%%%%%%%%%%%%%%%%%%%%%%%%%
%CITEVA DEFINITII
\def\om{\omega}
\def\Om{\Omega}
\def\et{\eta}
\def\td{\tilde{\delta}}
\def\m{{\mu}}
\def\n{{\nu}}
\def\k{{\kappa}}
\def\l{{\lambda}}
\def\L{{\Lambda}}
\def\g{{\gamma}}
\def\a{{\alpha}}
\def\e{{\varepsilon}}
\def\b{{\beta}}
\def\G{{\Gamma}}
\def\d{{\delta}}
\def\D{{\Delta}}
\def\T{{\Theta}}
\def\t{{\theta}}
\def\s{{\sigma}}
\def\S{{\Sigma}}
\def\z{{\zeta}}
\def\qed{\hfill\Box}
\def\ds{\displaystyle}
\def\mc{\mathcal}
%%%%%%%%%%%%%%%%%%%%%%%%%%%%%%%%%%%%%%%%%%%%%%%%%%%%%%%%%%%%%%%%%%%%%%%%%%%%%%%%%%%%%%%%%%%%%%%%%%%%%%%%%%%%%%%%%%%%%%
\def\1{{\mathbf 1}}
\def\CC{{\mathbb C}}
\def\RR{{\mathbb R}}
\def\QQ{{\mathbb Q}}
\def\ZZ{{\mathbb Z}}
\def\PP{{\mathbb P}}
\def\EE{{\mathbb E}}
\def\VV{{\mathbb V}}
\def\NN{{\mathbb N}}
\def\FF{{\mathbb F}}
%\def\SS{{\mathbb S}}
\def\MO{{\mathcal O}}
\def\MA{{\mathcal A}}
\def\MF{{\mathcal F}}
\def\MR{{\mathcal R}}
\def\MB{{\mathcal B}}
\def\MM{{\mathcal M}}
\def\MN{{\mathcal N}}
\def\MU{{\mathcal U}}
\def\MP{{\mathcal P}}
\def\MS{{\mathcal S}}
\def\MBS{{\mathbf S}}
\def\MX{{\bm{ \mathscr X}}}

% independent sign
\newcommand\independent{\protect\mathpalette{\protect\independenT}{\perp}}
\def\independenT#1#2{\mathrel{\rlap{$#1#2$}\mkern2mu{#1#2}}}

%%%%%%%%%%%%%%%%%%%%%%%%%%%%%%%%%%%%%%%%%%%%%%%%%%%%%%%%%%%%%%%%%%%%%%%%%%%%%%%%%%%%%%%%%%%%%%%%%%%%%%%%%%%%%%%%%%%%%
%Header and Footer
\usepackage{fancyhdr}

\pagestyle{fancy}
\fancyhf{}
\rhead{Universitatea din Bucure\c sti\\ Facultatea de Matematic\u a \c si Informatic\u a}
\lhead{\textit{Curs}: Statistic\u a (2017 - 2018)\\ \textit{Instructori}: A. Am\u arioarei, S. Cojocea}
\rfoot{Pagina \thepage}
\lfoot{Grupa: 321}
%%%%%%%%%%%%%%%%%%%%%%%%%%%%%%%%%%%%%%%
%%%%%%%%%%%%%%%%%%%%%%%%%%%%%%%%%%%%%%%
%%%%%%%%%%%%%%%%%%%%%%%%%%%%%%%%%%%%%%%
\usepackage{booktabs}
\usepackage{longtable}
\usepackage{framed,color}
\definecolor{shadecolor}{RGB}{248,248,248}

\ifxetex
  \usepackage{letltxmacro}
  \setlength{\XeTeXLinkMargin}{1pt}
  \LetLtxMacro\SavedIncludeGraphics\includegraphics
  \def\includegraphics#1#{% #1 catches optional stuff (star/opt. arg.)
    \IncludeGraphicsAux{#1}%
  }%
  \newcommand*{\IncludeGraphicsAux}[2]{%
    \XeTeXLinkBox{%
      \SavedIncludeGraphics#1{#2}%
    }%
  }%
\fi

\newenvironment{rmdblock}[1]
  {\begin{shaded*}
  \begin{itemize}
  \renewcommand{\labelitemi}{
    \raisebox{-.7\height}[0pt][0pt]{
      {\setkeys{Gin}{width=2em,keepaspectratio}\includegraphics{icons/#1}}
    }
  }
  \item
  }
  {
  \end{itemize}
  \end{shaded*}
  }
\newenvironment{rmdcaution}
  {\begin{rmdblock}{caution}}
  {\end{rmdblock}}
\newenvironment{rmdinsight}
  {\begin{rmdblock}{insight}}
  {\end{rmdblock}}
\newenvironment{rmdexercise}
  {\begin{rmdblock}{exercise}}
  {\end{rmdblock}}
\newenvironment{rmdtip}
  {\begin{rmdblock}{tip}}
  {\end{rmdblock}}
  
%%%%%%%%%%%%%%%%%%%%%%%%%%%%%%%%%%%%%%%
\usepackage{pifont}% http://ctan.org/pkg/pifont
\newcommand{\cmark}{\ding{51}}%
\newcommand{\xmark}{\ding{55}}%

\begin{document}
\maketitle

%%%%%%%%%%%%%%%%%%%%%%%%
\thispagestyle{fancy}

\begin{rmdcaution}
Timp de lucru 2h. Toate documentele, computerele personale, telefoanele
mobile și/sau calculatoarele electronice de mână sunt autorizate. Orice
modalitate de comunicare între voi este \textbf{strict interzisă}. Aveți
3 subiecte, fiecare valorând 10 puncte. Mult succes !
\end{rmdcaution}

\subsection{Exercițiul 1}\label{exercitiul-1}

Fie \(X\) o variabilă aleatoare repartizată

\[
  \mathbb{P}_{\theta}(X = k) = A(k+1)\theta^k,\quad k\in\mathbb{N}
\] unde \(\theta\in(0,1)\) un parametru necunoscut și \(A\in\mathbb{R}\)
este o constantă.

\begin{enumerate}
\def\labelenumi{\arabic{enumi}.}
\tightlist
\item
  Determinați constanta \(A\) și calculați \(\EE[X]\) și \(Var(X)\).
\end{enumerate}

Dorim să estimăm pe \(\theta\) plecând de la un eșantion
\(X_1, X_2, \ldots, X_n\) de talie \(n\) din populația dată de
repartiția lui \(X\).

\begin{enumerate}
\def\labelenumi{\arabic{enumi}.}
\setcounter{enumi}{1}
\item
  Determinați estimatorul \(\tilde{\theta}\) a lui \(\theta\) obținut
  prin metoda momentelor și calculați
  \(\mathbb{P}_{\theta}(\tilde{\theta} = 0)\).
\item
  Determinați estimatorul de verosimilitate maximă \(\hat{\theta}\) a
  lui \(\theta\) și verificați dacă acesta este bine definit.
\item
  Studiați consistența estimatorului \(\tilde{\theta}\) și determinați
  legea lui limită.
\end{enumerate}

\subsection{Exercițiul 2}\label{exercitiul-2}

Fie \(X_1, X_2, \ldots, X_n\) un eșantion de talie \(n\) din populația
\(f_{\theta}\) unde

\[
  f_{\theta}(x) = \frac{1}{\theta}e^{-\frac{x-\theta}{\theta}} \mathbf{1}_{[\theta, +\infty)}(x)
\]

cu \(\theta>0\), parametru necunoscut.

\begin{enumerate}
\def\labelenumi{\arabic{enumi}.}
\item
  \begin{enumerate}
  \def\labelenumii{\alph{enumii})}
  \item
    Determinați repartiția lui \(\frac{X_1}{\theta}-1\).
  \item
    Determinați estimatorul \(\tilde{\theta}\) a lui \(\theta\) obținut
    prin metoda momentelor și calculați eroarea pătratică medie a
    acestuia.
  \item
    Găsiți legea limită a lui \(\tilde{\theta}\).
  \end{enumerate}
\item
  \begin{enumerate}
  \def\labelenumii{\alph{enumii})}
  \item
    Determinați estimatorul \(\hat{\theta}\) a lui \(\theta\) obținut
    prin metoda verosimilității maxime.
  \item
    Calculați eroarea pătratică medie a lui \(\hat{\theta}\) și
    verificați dacă estimatorul este consistent.
  \item
    Construiți un interval de încredere pentru \(\theta\) de nivel de
    încredere \(1-\alpha\).
  \item
    Pe care dintre cei doi estimatori îl preferați ?
  \end{enumerate}
\end{enumerate}

\subsection{Exercițiul 3}\label{exercitiul-3}

Fie \(X_1, X_2, \ldots, X_n\) un eșantion de talie \(n\) din populația
\(f_{\theta}\) unde

\[
  f_{\theta}(x) = \frac{3}{(x-\theta)^4}\mathbf{1}_{[1+\theta,+\infty)}(x)
\]

\begin{enumerate}
\def\labelenumi{\arabic{enumi}.}
\item
  \begin{enumerate}
  \def\labelenumii{\alph{enumii})}
  \item
    Calculați \(\EE_{\theta}[X_1]\), \(Var_{\theta}(X_1)\) și funcția de
    repartiție \(F_{\theta}(x)\) a lui \(X_1\).
  \item
    În cazul în care \(\theta = 2\) dorim să generăm \(3\) valori
    aleatoare din repartiția lui \(X\sim f_{\theta}(x)\). Pentru aceasta
    dispunem de trei valori rezultate din repartiția uniformă pe {[}0,
    1{]} : \(u_1 = 0.25\), \(u_2 = 0.4\) și \(u_3 = 0.5\). Descrieți
    procedura.
  \end{enumerate}
\item
  \begin{enumerate}
  \def\labelenumii{\alph{enumii})}
  \item
    Determinați estimatorul \(\hat{\theta}_{n}^M\) a lui \(\theta\)
    obținut prin metoda momentelor și calculați eroarea pătratică medie
    a acestui estimator. Care este legea lui limită ?
  \item
    Găsiți un interval de încredere asimptotic de nivel de încredere de
    \(95\%\) pentru \(\theta\).
  \end{enumerate}
\item
  \begin{enumerate}
  \def\labelenumii{\alph{enumii})}
  \item
    Exprimați în funcție de \(\theta\) mediana repartiției lui \(X_1\)
    și plecând de la aceasta găsiți un alt estimator
    \(\hat{\theta}_{n}^Q\) al lui \(\theta\).
  \item
    Determinați legea lui limită a lui \(\hat{\theta}_{n}^Q\) și arătați
    că, asimptotic, acesta este mai bun decât \(\hat{\theta}_{n}^M\).
  \item
    Găsiți un interval de încredere asimptotic de nivel de încredere de
    \(95\%\) pentru \(\theta\).
  \end{enumerate}
\item
  \begin{enumerate}
  \def\labelenumii{\alph{enumii})}
  \item
    Determinați estimatorul de verosimilitate maximă
    \(\hat{\theta}_{n}^{VM}\) a lui \(\theta\) și verificați dacă este
    deplasat.
  \item
    Calculați funcția de repartiție a lui
    \(\hat{\theta}_{n}^{VM}-\theta\).
  \item
    Pe care dintre cei trei estimatori îl preferați ?
  \end{enumerate}
\end{enumerate}


\end{document}
