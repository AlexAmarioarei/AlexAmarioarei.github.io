\documentclass[]{article}
\usepackage{lmodern}
\usepackage{amssymb,amsmath}
\usepackage{ifxetex,ifluatex}
\usepackage{fixltx2e} % provides \textsubscript
\ifnum 0\ifxetex 1\fi\ifluatex 1\fi=0 % if pdftex
  \usepackage[T1]{fontenc}
  \usepackage[utf8]{inputenc}
\else % if luatex or xelatex
  \ifxetex
    \usepackage{mathspec}
  \else
    \usepackage{fontspec}
  \fi
  \defaultfontfeatures{Ligatures=TeX,Scale=MatchLowercase}
\fi
% use upquote if available, for straight quotes in verbatim environments
\IfFileExists{upquote.sty}{\usepackage{upquote}}{}
% use microtype if available
\IfFileExists{microtype.sty}{%
\usepackage{microtype}
\UseMicrotypeSet[protrusion]{basicmath} % disable protrusion for tt fonts
}{}
\usepackage[margin=1in]{geometry}
\usepackage{hyperref}
\hypersetup{unicode=true,
            pdftitle={Proiect},
            pdfborder={0 0 0},
            breaklinks=true}
\urlstyle{same}  % don't use monospace font for urls
\usepackage{graphicx,grffile}
\makeatletter
\def\maxwidth{\ifdim\Gin@nat@width>\linewidth\linewidth\else\Gin@nat@width\fi}
\def\maxheight{\ifdim\Gin@nat@height>\textheight\textheight\else\Gin@nat@height\fi}
\makeatother
% Scale images if necessary, so that they will not overflow the page
% margins by default, and it is still possible to overwrite the defaults
% using explicit options in \includegraphics[width, height, ...]{}
\setkeys{Gin}{width=\maxwidth,height=\maxheight,keepaspectratio}
\IfFileExists{parskip.sty}{%
\usepackage{parskip}
}{% else
\setlength{\parindent}{0pt}
\setlength{\parskip}{6pt plus 2pt minus 1pt}
}
\setlength{\emergencystretch}{3em}  % prevent overfull lines
\providecommand{\tightlist}{%
  \setlength{\itemsep}{0pt}\setlength{\parskip}{0pt}}
\setcounter{secnumdepth}{5}
% Redefines (sub)paragraphs to behave more like sections
\ifx\paragraph\undefined\else
\let\oldparagraph\paragraph
\renewcommand{\paragraph}[1]{\oldparagraph{#1}\mbox{}}
\fi
\ifx\subparagraph\undefined\else
\let\oldsubparagraph\subparagraph
\renewcommand{\subparagraph}[1]{\oldsubparagraph{#1}\mbox{}}
\fi

%%% Use protect on footnotes to avoid problems with footnotes in titles
\let\rmarkdownfootnote\footnote%
\def\footnote{\protect\rmarkdownfootnote}

%%% Change title format to be more compact
\usepackage{titling}

% Create subtitle command for use in maketitle
\newcommand{\subtitle}[1]{
  \posttitle{
    \begin{center}\large#1\end{center}
    }
}

\setlength{\droptitle}{-2em}
  \title{Proiect}
  \pretitle{\vspace{\droptitle}\centering\huge}
  \posttitle{\par}
\subtitle{Grupa 301}
  \author{}
  \preauthor{}\postauthor{}
  \date{}
  \predate{}\postdate{}

\usepackage{booktabs}
\usepackage{longtable}
\usepackage{framed,color}
\definecolor{shadecolor}{RGB}{248, 248, 248}
\definecolor{shadecolor1}{RGB}{216,225,235}
\definecolor{framecolor}{RGB}{108,123,13}

\ifxetex
  \usepackage{letltxmacro}
  \setlength{\XeTeXLinkMargin}{1pt}
  \LetLtxMacro\SavedIncludeGraphics\includegraphics
  \def\includegraphics#1#{% #1 catches optional stuff (star/opt. arg.)
    \IncludeGraphicsAux{#1}%
  }%
  \newcommand*{\IncludeGraphicsAux}[2]{%
    \XeTeXLinkBox{%
      \SavedIncludeGraphics#1{#2}%
    }%
  }%
\fi

\newenvironment{frshaded*}{%
  \def\FrameCommand{\fboxrule=\FrameRule\fboxsep=\FrameSep \fcolorbox{framecolor}{shadecolor1}}%
  \MakeFramed {\advance\hsize-\width \FrameRestore}}%
{\endMakeFramed}

\newenvironment{rmdblock}[1]
  {\begin{frshaded*}
  \begin{itemize}
  \renewcommand{\labelitemi}{
    \raisebox{-.7\height}[0pt][0pt]{
      {\setkeys{Gin}{width=2em,keepaspectratio}\includegraphics{images/icons/#1}}
    }
  }
  \item
  }
  {
  \end{itemize}
  \end{frshaded*}
  }

\newenvironment{rmdcaution}
  {\begin{rmdblock}{caution}}
  {\end{rmdblock}}
% \newenvironment{rmdinsight}
%   {\begin{rmdblock}{insight}}
%   {\end{rmdblock}}
\newenvironment{rmdexercise}
  {\begin{rmdblock}{exercise}}
  {\end{rmdblock}}
\newenvironment{rmdtip}
  {\begin{rmdblock}{tip}}
  {\end{rmdblock}}


%%%%%%%%%%%%%%%%%%%%%%%%%%%%%%%%%%%%%%%%%%%%%%%%%%%%%%%%%%%%%%%%%%%%%%%%%%%%%%%%%%%%%%%%%%%%%%%%%%%%%%%%%%%%%%%%%%%%%
%%%%%%%%%%% For insight block %%%%%%%%%%%%%%%%%%%%%%%%%%
\definecolor{shadecolor_insight}{RGB}{223,240,216}
\definecolor{framecolor_insight}{RGB}{136,193,137}

\newenvironment{frshaded_insight*}{%
  \def\FrameCommand{\fboxrule=\FrameRule\fboxsep=\FrameSep \fcolorbox{framecolor_insight}{shadecolor_insight}}%
  \MakeFramed {\advance\hsize-\width \FrameRestore}}%
{\endMakeFramed}

\newenvironment{rmdblock_insight}[1]
  {\begin{frshaded_insight*}
  \begin{itemize}
  \renewcommand{\labelitemi}{
    \raisebox{-.7\height}[0pt][0pt]{
      {\setkeys{Gin}{width=2em,keepaspectratio}\includegraphics{images/icons/#1}}
    }
  }
  \item
  }
  {
  \end{itemize}
  \end{frshaded_insight*}
  }

\newenvironment{rmdinsight}
  {\begin{rmdblock_insight}{insight}}
  {\end{rmdblock_insight}}

%%%%%%%%%%%%%%%%%%%%%%%%%%%%%%%%%%%%%%%%%%%%%%%%%%%%%%%%%%%%%%%%%%%%%%%%%%%%%%%%%%%%%%%%%%%%%%%%%%%%%%%%%%%%%%%%%%%%%
\usepackage{subfigure}
\usepackage{booktabs}
\usepackage{slashbox}
\usepackage{color}
%%%%%%%%%%%%%%%%%%%%%%%%%%%%%%%%%%%%%%%%%%%%%%%%%%%%%%%%%%%%%%%%%%%%%%%%%%%%%%%%%%%%%%%%%%%%%%%%%%%%%%%%%%%%%%%%%%%%%
%CITEVA DEFINITII
\def\om{\omega}
\def\Om{\Omega}
\def\et{\eta}
\def\td{\tilde{\delta}}
\def\m{{\mu}}
\def\n{{\nu}}
\def\k{{\kappa}}
\def\l{{\lambda}}
\def\L{{\Lambda}}
\def\g{{\gamma}}
\def\a{{\alpha}}
\def\e{{\varepsilon}}
\def\b{{\beta}}
\def\G{{\Gamma}}
\def\d{{\delta}}
\def\D{{\Delta}}
\def\t{{\theta}}
\def\s{{\sigma}}
\def\S{{\Sigma}}
\def\z{{\zeta}}
\def\qed{\hfill\Box}
\def\ds{\displaystyle}
\def\mc{\mathcal}
%%%%%%%%%%%%%%%%%%%%%%%%%%%%%%%%%%%%%%%%%%%%%%%%%%%%%%%%%%%%%%%%%%%%%%%%%%%%%%%%%%%%%%%%%%%%%%%%%%%%%%%%%%%%%%%%%%%%%%
\def\1{{\mathbf 1}}
\def\CC{{\mathbb C}}
\def\VV{{\mathbb V}}
\def\RR{{\mathbb R}}
\def\QQ{{\mathbb Q}}
\def\ZZ{{\mathbb Z}}
\def\PP{{\mathbb P}}
\def\EE{{\mathbb E}}
\def\NN{{\mathbb N}}
\def\FF{{\mathbb F}}
%\def\SS{{\mathbb S}}
\def\MA{{\mathcal A}}
\def\MO{{\mathcal O}}
\def\MF{{\mathcal F}}
\def\ME{{\mathcal E}}
\def\MR{{\mathcal R}}
\def\MB{{\mathcal B}}
\def\MM{{\mathcal M}}
\def\MN{{\mathcal N}}
\def\MU{{\mathcal U}}
\def\MP{{\mathcal P}}
\def\MS{{\mathcal S}}
\def\MBS{{\mathbf S}}
\def\MX{{\bm{ \mathscr X}}}

% independent sign
\newcommand\independent{\protect\mathpalette{\protect\independenT}{\perp}}
\def\independenT#1#2{\mathrel{\rlap{$#1#2$}\mkern2mu{#1#2}}}

%%%%%%%%%%%%%%%%%%%%%%%%%%%%%%%%%%%%%%%%%%%%%%%%%%%%%%%%%%%%%%%%%%%%%%%%%%%%%%%%%%%%%%%%%%%%%%%%%%%%%%%%%%%%%%%%%%%%%
%Header and Footer
\usepackage{fancyhdr}

\pagestyle{fancy}
\fancyhf{}
\rhead{Universitatea din Bucure\c sti\\ Facultatea de Matematic\u a \c si Informatic\u a}
\lhead{\textit{Curs}: Statistic\u a\\ \textit{Instructor}: A. Am\u arioarei, S. Cojocea}
\rfoot{Pagina \thepage}
\lfoot{Grupele: 301, 311, 321}
%%%%%%%%%%%%%%%%%%%%%%%%%%%%%%%%%%%%%%%
\usepackage{booktabs}
\usepackage{longtable}
\usepackage{array}
\usepackage{multirow}
\usepackage[table]{xcolor}
\usepackage{wrapfig}
\usepackage{float}
\usepackage{colortbl}
\usepackage{pdflscape}
\usepackage{tabu}
\usepackage{threeparttable}

\begin{document}
\maketitle

%%%%%%%%%%%%%%%%%%%%%%%%
\thispagestyle{fancy}

\textbf{Notă:} Raportul poate fi scris în \emph{Microsoft Word} sau
\LaTeX (pentru ușurință recomand folosirea pachetului \emph{rmarkdown}
din \emph{R} - mai multe informații găsiți pe site la secțiunea
\emph{Link-uri utile}). Toate simulările, figurile și codurile folosite
trebuie incluse în raport. Se va folosi doar limbajul \emph{R}.

\section{Problemă}\label{problema}

Un segment de lungime \(1\) este rupt în trei bucăți. Presupunând că
punctele de ruptură sunt date de două variabile aleatoare \(X\) și \(Y\)
repartizate pe \([0,1]\), Scopul acestui exercțiu este să determinăm, în
funcție de procedura de alegere a punctelor de ruptură, care este
probabilitatea de formare a unui triunghi cu lungimile celor trei
segmente obținute.

\textbf{Procedura \(1\):} Presupunem că punctele de ruptură sunt date de
două variabile aleatoare \(X\) și \(Y\), independente și repartizate
uniform pe intervalul \([0,1]\).

\begin{enumerate}
\def\labelenumi{\arabic{enumi}.}
\item
  Fie \(a\), \(b\) și \(c\) lungimile celor trei segmente obținute
  (luate de la stanga la dreapta). Arătați că lungimile celor trei
  segmente pot forma un triunghi dacă și numai dacă fiecare dintre cele
  trei lungimi este mai mică sau egală cu \(\frac{1}{2}\). Traduceți
  această condiție in funcție de v.a. \(X\) și \(Y\).
\item
  Într-o primă etapă dorim să aproximăm lungimile medii ale celor trei
  segmente obținute. Pentru aceasta, simulăm \(N=5000\) de realizări
  independente ale cuplului \((X,Y)\). Care sunt valoriile lungimilor
  medii ? Ce teoremă limită justifică acest rezultat ?
\item
  Dorim de asemenea să răspundem într-o manieră mai fină la intrebarea
  problemei:

  \begin{enumerate}
  \def\labelenumii{\alph{enumii}.}
  \item
    Cuplul de puncte \((X,Y)\) de ruptură poate fi văzut ca un punct în
    pătratul unitate \([0,1]^2\). Plecând de la \(5000\) de realizări
    independente ale cuplului \((X,Y)\), reprezentați grafic punctele
    \((X_i,Y_i)\) din interiorul pătratului \([0,1]^2\) care determină
    cele trei segmente cu ajutorul cărora putem forma un triunghi, cu
    albastru, și pe celelalte cu roșu.
  \item
    Plecând de la \(N=5000\) de realizări independente ale cuplului
    \((X,Y)\), estimați probabilitatea căutată.
  \item
    Găsiți această probabilitate teoretic și comparați cu rezultatul
    găsit la punctul anterior.
  \end{enumerate}
\end{enumerate}

4.\footnote{Această întrebare este mai dificilă și nerezolvarea ei nu
  scade punctajul proiectului. Cu toate acestea, cine reușește să facă
  acest subpunct va primi 0.5 puncte suplimentare.} Presupunând că
punctele de ruptură sunt date de procedura \(1\), ce puteți spune despre
probabilitatea de formare a unui triunghi obtuzunghic ? Justificați atât
teoretic cât și prin simulare.

Ne întrebăm acum ce se întâmplă cu probabilitatea de formare a unui
triunghi cu ajutorul celor trei segmente determinate de punctele de
ruptură dacă adoptăm următoarele două proceduri.

\textbf{Procedura 2:} Alegem pentru început un punct de ruptură \(X\)
repartizat uniform \(\mathcal{U}([0,1])\) și dintre cele două segmente
formate îl alegem pe cel mai lung pe care alegem un al doilea punct,
\(Y\), repartizat uniform pe acest segment.

\begin{enumerate}
\def\labelenumi{\arabic{enumi}.}
\setcounter{enumi}{4}
\item
  Reconsiderați punctele b. și c. de la punctul 3.
\item
  Pentru a ilustra grafic regiunea determinată de perechile care
  verifică problema (formează un triunghi) putem considera variabila
  aleatoare \(Z\) repartizată uniform pe intervalul \([0,1]\) și care să
  fie independentă de \(X\) și să exprimăm lungimile celor trei segmente
  în funcție de aceasta. Plecând de la \(5000\) de realizări
  independente ale cuplului \((X,Z)\), reprezentați grafic, cu albastru,
  punctele \((X_i, Z_i)\) din interiorul pătratului \([0,1]\) care
  determină cele trei segmente cu ajutorul cărora putem forma un
  triunghi și pe celelalte cu roșu.
\end{enumerate}

\textbf{Procedura 3:} Alegem pentru început un punct de ruptură \(X\)
repartizat uniform \(\mathcal{U}([0,1])\) care va împărții segmentul
\([0,1]\) în două subsegmente. Aruncăm, de manieră independentă, o
monedă echilibrată și alegem în funcție de rezultatul aruncării, cap sau
pajură, segementul din stânga (cap) sau cel din dreapta (pajură). Pe
segmentul selecționat alegem un al doilea punct de ruptură, \(Y\),
repartizat uniform pe acest segment.

\begin{enumerate}
\def\labelenumi{\arabic{enumi}.}
\setcounter{enumi}{6}
\tightlist
\item
  Reconsiderați punctele 5. și 6. de la \emph{Procedura 2}.
\end{enumerate}


\end{document}
