\documentclass[]{article}
\usepackage{lmodern}
\usepackage{amssymb,amsmath}
\usepackage{ifxetex,ifluatex}
\usepackage{fixltx2e} % provides \textsubscript
\ifnum 0\ifxetex 1\fi\ifluatex 1\fi=0 % if pdftex
  \usepackage[T1]{fontenc}
  \usepackage[utf8]{inputenc}
\else % if luatex or xelatex
  \ifxetex
    \usepackage{mathspec}
  \else
    \usepackage{fontspec}
  \fi
  \defaultfontfeatures{Ligatures=TeX,Scale=MatchLowercase}
\fi
% use upquote if available, for straight quotes in verbatim environments
\IfFileExists{upquote.sty}{\usepackage{upquote}}{}
% use microtype if available
\IfFileExists{microtype.sty}{%
\usepackage{microtype}
\UseMicrotypeSet[protrusion]{basicmath} % disable protrusion for tt fonts
}{}
\usepackage[margin=1in]{geometry}
\usepackage{hyperref}
\hypersetup{unicode=true,
            pdftitle={Proiect},
            pdfborder={0 0 0},
            breaklinks=true}
\urlstyle{same}  % don't use monospace font for urls
\usepackage{graphicx,grffile}
\makeatletter
\def\maxwidth{\ifdim\Gin@nat@width>\linewidth\linewidth\else\Gin@nat@width\fi}
\def\maxheight{\ifdim\Gin@nat@height>\textheight\textheight\else\Gin@nat@height\fi}
\makeatother
% Scale images if necessary, so that they will not overflow the page
% margins by default, and it is still possible to overwrite the defaults
% using explicit options in \includegraphics[width, height, ...]{}
\setkeys{Gin}{width=\maxwidth,height=\maxheight,keepaspectratio}
\IfFileExists{parskip.sty}{%
\usepackage{parskip}
}{% else
\setlength{\parindent}{0pt}
\setlength{\parskip}{6pt plus 2pt minus 1pt}
}
\setlength{\emergencystretch}{3em}  % prevent overfull lines
\providecommand{\tightlist}{%
  \setlength{\itemsep}{0pt}\setlength{\parskip}{0pt}}
\setcounter{secnumdepth}{5}
% Redefines (sub)paragraphs to behave more like sections
\ifx\paragraph\undefined\else
\let\oldparagraph\paragraph
\renewcommand{\paragraph}[1]{\oldparagraph{#1}\mbox{}}
\fi
\ifx\subparagraph\undefined\else
\let\oldsubparagraph\subparagraph
\renewcommand{\subparagraph}[1]{\oldsubparagraph{#1}\mbox{}}
\fi

%%% Use protect on footnotes to avoid problems with footnotes in titles
\let\rmarkdownfootnote\footnote%
\def\footnote{\protect\rmarkdownfootnote}

%%% Change title format to be more compact
\usepackage{titling}

% Create subtitle command for use in maketitle
\newcommand{\subtitle}[1]{
  \posttitle{
    \begin{center}\large#1\end{center}
    }
}

\setlength{\droptitle}{-2em}
  \title{Proiect}
  \pretitle{\vspace{\droptitle}\centering\huge}
  \posttitle{\par}
\subtitle{Grupele 311, 321}
  \author{}
  \preauthor{}\postauthor{}
  \date{}
  \predate{}\postdate{}

\usepackage{booktabs}
\usepackage{longtable}
\usepackage{framed,color}
\definecolor{shadecolor}{RGB}{248, 248, 248}
\definecolor{shadecolor1}{RGB}{216,225,235}
\definecolor{framecolor}{RGB}{108,123,13}

\ifxetex
  \usepackage{letltxmacro}
  \setlength{\XeTeXLinkMargin}{1pt}
  \LetLtxMacro\SavedIncludeGraphics\includegraphics
  \def\includegraphics#1#{% #1 catches optional stuff (star/opt. arg.)
    \IncludeGraphicsAux{#1}%
  }%
  \newcommand*{\IncludeGraphicsAux}[2]{%
    \XeTeXLinkBox{%
      \SavedIncludeGraphics#1{#2}%
    }%
  }%
\fi

\newenvironment{frshaded*}{%
  \def\FrameCommand{\fboxrule=\FrameRule\fboxsep=\FrameSep \fcolorbox{framecolor}{shadecolor1}}%
  \MakeFramed {\advance\hsize-\width \FrameRestore}}%
{\endMakeFramed}

\newenvironment{rmdblock}[1]
  {\begin{frshaded*}
  \begin{itemize}
  \renewcommand{\labelitemi}{
    \raisebox{-.7\height}[0pt][0pt]{
      {\setkeys{Gin}{width=2em,keepaspectratio}\includegraphics{images/icons/#1}}
    }
  }
  \item
  }
  {
  \end{itemize}
  \end{frshaded*}
  }

\newenvironment{rmdcaution}
  {\begin{rmdblock}{caution}}
  {\end{rmdblock}}
% \newenvironment{rmdinsight}
%   {\begin{rmdblock}{insight}}
%   {\end{rmdblock}}
\newenvironment{rmdexercise}
  {\begin{rmdblock}{exercise}}
  {\end{rmdblock}}
\newenvironment{rmdtip}
  {\begin{rmdblock}{tip}}
  {\end{rmdblock}}


%%%%%%%%%%%%%%%%%%%%%%%%%%%%%%%%%%%%%%%%%%%%%%%%%%%%%%%%%%%%%%%%%%%%%%%%%%%%%%%%%%%%%%%%%%%%%%%%%%%%%%%%%%%%%%%%%%%%%
%%%%%%%%%%% For insight block %%%%%%%%%%%%%%%%%%%%%%%%%%
\definecolor{shadecolor_insight}{RGB}{223,240,216}
\definecolor{framecolor_insight}{RGB}{136,193,137}

\newenvironment{frshaded_insight*}{%
  \def\FrameCommand{\fboxrule=\FrameRule\fboxsep=\FrameSep \fcolorbox{framecolor_insight}{shadecolor_insight}}%
  \MakeFramed {\advance\hsize-\width \FrameRestore}}%
{\endMakeFramed}

\newenvironment{rmdblock_insight}[1]
  {\begin{frshaded_insight*}
  \begin{itemize}
  \renewcommand{\labelitemi}{
    \raisebox{-.7\height}[0pt][0pt]{
      {\setkeys{Gin}{width=2em,keepaspectratio}\includegraphics{images/icons/#1}}
    }
  }
  \item
  }
  {
  \end{itemize}
  \end{frshaded_insight*}
  }

\newenvironment{rmdinsight}
  {\begin{rmdblock_insight}{insight}}
  {\end{rmdblock_insight}}

%%%%%%%%%%%%%%%%%%%%%%%%%%%%%%%%%%%%%%%%%%%%%%%%%%%%%%%%%%%%%%%%%%%%%%%%%%%%%%%%%%%%%%%%%%%%%%%%%%%%%%%%%%%%%%%%%%%%%
\usepackage{subfigure}
\usepackage{booktabs}
\usepackage{slashbox}
\usepackage{color}
%%%%%%%%%%%%%%%%%%%%%%%%%%%%%%%%%%%%%%%%%%%%%%%%%%%%%%%%%%%%%%%%%%%%%%%%%%%%%%%%%%%%%%%%%%%%%%%%%%%%%%%%%%%%%%%%%%%%%
%CITEVA DEFINITII
\def\om{\omega}
\def\Om{\Omega}
\def\et{\eta}
\def\td{\tilde{\delta}}
\def\m{{\mu}}
\def\n{{\nu}}
\def\k{{\kappa}}
\def\l{{\lambda}}
\def\L{{\Lambda}}
\def\g{{\gamma}}
\def\a{{\alpha}}
\def\e{{\varepsilon}}
\def\b{{\beta}}
\def\G{{\Gamma}}
\def\d{{\delta}}
\def\D{{\Delta}}
\def\t{{\theta}}
\def\s{{\sigma}}
\def\S{{\Sigma}}
\def\z{{\zeta}}
\def\qed{\hfill\Box}
\def\ds{\displaystyle}
\def\mc{\mathcal}
%%%%%%%%%%%%%%%%%%%%%%%%%%%%%%%%%%%%%%%%%%%%%%%%%%%%%%%%%%%%%%%%%%%%%%%%%%%%%%%%%%%%%%%%%%%%%%%%%%%%%%%%%%%%%%%%%%%%%%
\def\1{{\mathbf 1}}
\def\CC{{\mathbb C}}
\def\VV{{\mathbb V}}
\def\RR{{\mathbb R}}
\def\QQ{{\mathbb Q}}
\def\ZZ{{\mathbb Z}}
\def\PP{{\mathbb P}}
\def\EE{{\mathbb E}}
\def\NN{{\mathbb N}}
\def\FF{{\mathbb F}}
%\def\SS{{\mathbb S}}
\def\MA{{\mathcal A}}
\def\MO{{\mathcal O}}
\def\MF{{\mathcal F}}
\def\ME{{\mathcal E}}
\def\MR{{\mathcal R}}
\def\MB{{\mathcal B}}
\def\MM{{\mathcal M}}
\def\MN{{\mathcal N}}
\def\MU{{\mathcal U}}
\def\MP{{\mathcal P}}
\def\MS{{\mathcal S}}
\def\MBS{{\mathbf S}}
\def\MX{{\bm{ \mathscr X}}}

% independent sign
\newcommand\independent{\protect\mathpalette{\protect\independenT}{\perp}}
\def\independenT#1#2{\mathrel{\rlap{$#1#2$}\mkern2mu{#1#2}}}

%%%%%%%%%%%%%%%%%%%%%%%%%%%%%%%%%%%%%%%%%%%%%%%%%%%%%%%%%%%%%%%%%%%%%%%%%%%%%%%%%%%%%%%%%%%%%%%%%%%%%%%%%%%%%%%%%%%%%
%Header and Footer
\usepackage{fancyhdr}

\pagestyle{fancy}
\fancyhf{}
\rhead{Universitatea din Bucure\c sti\\ Facultatea de Matematic\u a \c si Informatic\u a}
\lhead{\textit{Curs}: Statistic\u a\\ \textit{Instructor}: A. Am\u arioarei, S. Cojocea}
\rfoot{Pagina \thepage}
\lfoot{Grupele: 301, 311, 321}
%%%%%%%%%%%%%%%%%%%%%%%%%%%%%%%%%%%%%%%
\usepackage{booktabs}
\usepackage{longtable}
\usepackage{array}
\usepackage{multirow}
\usepackage[table]{xcolor}
\usepackage{wrapfig}
\usepackage{float}
\usepackage{colortbl}
\usepackage{pdflscape}
\usepackage{tabu}
\usepackage{threeparttable}

\begin{document}
\maketitle

%%%%%%%%%%%%%%%%%%%%%%%%
\thispagestyle{fancy}

\textbf{Notă:} Rezolvarea problemelor de mai jos va fi realizată în
\textbf{R} (scripturile trebuie să fie comentate) şi va fi însoţită de
un document text (.pdf sau .docx) care să conţină comentarii şi
concluzii, acolo unde sunt cerute.

\textbf{Punctaj:} 1. 0.5p , 2. 0.75p, 3. 0.25p 4. 0.25p 5. 0.25p
\textbf{BONUS:} 0.5 p

\section{Problemă}\label{problema}

\begin{enumerate}
\def\labelenumi{\arabic{enumi}.}
\item
  Generaţi \(10 000\) de variabile aleatoare folosind \textbf{metoda
  transformării inverse} pentru repartiţiile definite mai jos:

  \begin{enumerate}
  \def\labelenumii{\alph{enumii})}
  \item
    Repartiţia logistică are densitatea de probabilitate
    \(f(x) = \frac{e^{-\frac{x-\mu}{\beta}}}{\beta\left(1+e^{-\frac{x-\mu}{\beta}}\right)^2}\)
    și funcția de repartiție
    \(F(x) = \frac{1}{1+e^{-\frac{x-\mu}{\beta}}}\).
  \item
    Repartiţia Cauchy are densitatea de probabilitate
    \(f(x) = \frac{1}{\pi \sigma}\frac{1}{1+\left(\frac{x-\mu}{\sigma}\right)^2}\)
    și funcția de repartiție
    \(F(x) = \frac{1}{2}+\frac{1}{\pi}\arctan\left(\frac{x-\mu}{\sigma}\right)\).
  \end{enumerate}
\end{enumerate}

Comparaţi rezultatele obţinute cu valorile date de funcţiile
\texttt{rlogis} şi respectiv \texttt{rcauchy} (funcţiile de repartiţie
predefinite în R pentru repartiţiile logistică şi respectiv Cauchy).
Ilustraţi grafic aceste rezultate.

\begin{enumerate}
\def\labelenumi{\arabic{enumi}.}
\setcounter{enumi}{1}
\item
  Folosiţi \textbf{metoda respingerii} pentru a genera observaţii din
  densitatea de probabilitate definită prin
  \(f(x)\propto e^{-\frac{x^2}{2}}\left[\sin(6x)^2+3\cos(x)^2\sin(4x)^2+1\right]\)\footnote{Notația
    \(\propto\) înseamnă că \(f(x)\) este proporțională cu expresia din
    dreapta} parcurgând paşii următori:

  \begin{enumerate}
  \def\labelenumii{\alph{enumii})}
  \item
    Reprezentaţi grafic \(f(x)\) şi arătaţi că aceasta este mărginită de
    \(Mg(x)\) unde \(g(x)\) este densitatea de probabilitate a
    repartiţiei normale standard. Determinaţi o valoare potrivită pentru
    constanta \(M\), chiar dacă nu este optimă\footnote{\textbf{Indiciu:}
      Folosiţi funcţia \texttt{optimise} din R}.
  \item
    Generaţi \(2500\) de observaţii din densitatea de mai sus folosind
    metoda respingerii.
  \item
    Deduceţi, pornind de la rata de acceptare a acestui algoritm, o
    aproximare a \emph{constantei de normalizare} a lui \(f(x)\), apoi
    comparaţi histograma valorilor generate cu reprezentarea grafică a
    lui \(f(x)\) normalizată.
  \end{enumerate}
\item
  \textbf{Metoda Monte Carlo pentru aproximarea unor integrale}
\end{enumerate}

Punctul de plecare al metodei Monte Carlo pentru aproximarea unei
integrale este nevoia de a evalua expresia
\(\mathbb{E}_{f}[h(X)] = \int_{\chi}h(x)f(x)\,dx\), unde \(\chi\)
reprezintă mulţimea de valori a variabile aleatoare \(X\) (care este, de
obicei, suportul densităţii \(f\)).

Principiul metodei Monte Carlo este de a aproxima expresia de mai sus cu
media de selecţie \(\bar{h}_n = \frac{1}{n}\sum_{j = 1}^{n}h(X_j)\)
pornind de la un eşantion \(X_1,X_2,\ldots,X_n\) din densitatea \(f\),
întrucât aceasta converge a.s. către \(\mathbb{E}_{f}[h(X)]\), conform
legii numerelor mari. Mai mult, atunci când \(h(X)^2\) are medie finită
viteza de convergenţă a lui \(\bar{h}_n\) poate fi determinată întrucât
convergenţa este de ordin \(\mathcal{O}(\sqrt{n})\) iar varianţa
aproximării este
\(Var(\bar{h}_n) = \frac{1}{n}\int_{\chi}\left(h(x) - \mathbb{E}_{f}[h(X)]\right)^2f(x)\,dx\),
cantitate care poate fi de asemenea aproximată prin
\(v_n = \frac{1}{n^2}\sum_{j=1}^{n}\left(h(X_j) - \bar{h}_n\right)^2\).

Mai precis, datorită teoremei limită centrală, pentru un \(n\) suficient
de mare expresia

\[
\frac{\bar{h}_n - \mathbb{E}_{f}[h(X)]}{\sqrt{v_n}}
\]

poate fi aproximată cu o normală standard, ceea ce conduce la
posibilitatea construirii unui test de convergenţă şi a unor margini
pentru aproximarea lui \(\mathbb{E}_{f}[h(X)]\).

\textbf{Cerință:}

Pentru funcţia \(h(x) = \left(\cos(50x) + \sin(20x)\right)^2\)
construiţi o aproximare a integralei acesteia pe intervalul {[}0,1{]}
după cum urmează: valoarea integralei poate fi văzută ca fiind media
funcţiei \(h(X)\) unde \(X\) este repartizată uniform pe \([0,1]\).
Urmărind algoritmul dat de metoda Monte Carlo construiţi programul R
care determină aproximarea acestei integrale. Comparaţi rezultatul
obţinut cu cel analitic și cu cel numeric obținut folosind funcția
\texttt{integrate}. Ataşaţi reprezentările grafice pe care le
consideraţi utile pentru a putea observa eficienţa metodei.

\begin{enumerate}
\def\labelenumi{\arabic{enumi}.}
\setcounter{enumi}{3}
\item
  Pornind de la un set de date din cele oferite de R (\emph{fiecare
  student îşi alege singur setul de date}) realizaţi o analiză de tipul
  ``statistică descriptivă'' a acestora (medie, cuartile, histogramă,
  boxplot, boxplot comparativ, identificare de outlieri, etc.).
  Documentaţi corespunzător acest proces şi explicaţi ce concluzii
  puteţi trage în urma acestei analize asupra datelor. Includeţi în
  fişier şi o descriere a datelor şi sursa lor.
\item
  Reprezentaţi grafic densitatea de probabilitate şi funcţia de
  repartiţie (în câte două grafice alăturate) pentru un set de parametri
  la alegere (cel puţin 4 cu reprezentările realizate în acelaşi grafic)
  pentru repartiţiile \textbf{Student}, \textbf{Fisher} şi \(\chi^2\).
  Identificaţi proprietăţile lor şi daţi un exemplu, construind un
  program în R, de utilizarea lor în testarea unor ipoteze statistice.
\end{enumerate}

\textbf{BONUS:}

\begin{enumerate}
\def\labelenumi{\arabic{enumi}.}
\setcounter{enumi}{5}
\tightlist
\item
  Construiţi un script R util în unul din experimentele de Machine
  Learning disponibile aici: \url{https://studio.azureml.net/} (vezi
  exemplu la ultimul laborator !!!)
\end{enumerate}


\end{document}
