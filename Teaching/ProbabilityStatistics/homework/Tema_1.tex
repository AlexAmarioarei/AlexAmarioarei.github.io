\documentclass[]{article}
\usepackage{lmodern}
\usepackage{amssymb,amsmath}
\usepackage{ifxetex,ifluatex}
\usepackage{fixltx2e} % provides \textsubscript
\ifnum 0\ifxetex 1\fi\ifluatex 1\fi=0 % if pdftex
  \usepackage[T1]{fontenc}
  \usepackage[utf8]{inputenc}
\else % if luatex or xelatex
  \ifxetex
    \usepackage{mathspec}
  \else
    \usepackage{fontspec}
  \fi
  \defaultfontfeatures{Ligatures=TeX,Scale=MatchLowercase}
\fi
% use upquote if available, for straight quotes in verbatim environments
\IfFileExists{upquote.sty}{\usepackage{upquote}}{}
% use microtype if available
\IfFileExists{microtype.sty}{%
\usepackage{microtype}
\UseMicrotypeSet[protrusion]{basicmath} % disable protrusion for tt fonts
}{}
\usepackage[margin=1in]{geometry}
\usepackage{hyperref}
\hypersetup{unicode=true,
            pdftitle={Tema 1},
            pdfborder={0 0 0},
            breaklinks=true}
\urlstyle{same}  % don't use monospace font for urls
\usepackage{graphicx,grffile}
\makeatletter
\def\maxwidth{\ifdim\Gin@nat@width>\linewidth\linewidth\else\Gin@nat@width\fi}
\def\maxheight{\ifdim\Gin@nat@height>\textheight\textheight\else\Gin@nat@height\fi}
\makeatother
% Scale images if necessary, so that they will not overflow the page
% margins by default, and it is still possible to overwrite the defaults
% using explicit options in \includegraphics[width, height, ...]{}
\setkeys{Gin}{width=\maxwidth,height=\maxheight,keepaspectratio}
\IfFileExists{parskip.sty}{%
\usepackage{parskip}
}{% else
\setlength{\parindent}{0pt}
\setlength{\parskip}{6pt plus 2pt minus 1pt}
}
\setlength{\emergencystretch}{3em}  % prevent overfull lines
\providecommand{\tightlist}{%
  \setlength{\itemsep}{0pt}\setlength{\parskip}{0pt}}
\setcounter{secnumdepth}{0}
% Redefines (sub)paragraphs to behave more like sections
\ifx\paragraph\undefined\else
\let\oldparagraph\paragraph
\renewcommand{\paragraph}[1]{\oldparagraph{#1}\mbox{}}
\fi
\ifx\subparagraph\undefined\else
\let\oldsubparagraph\subparagraph
\renewcommand{\subparagraph}[1]{\oldsubparagraph{#1}\mbox{}}
\fi
%%%%%%%%%%%%%%%%%%%%%%%%%%%%%%%%%%%%%%%%%%%%%%%%%%%%%%%%%%%%%%%%%%%%%%%%%%%%%%%%%%%%%%%%%%%%%%%%%%%%%%%%%%%%%%%%%%%%%
%%%%%%%%%%%%%%%%%%%%%%%%%%%%%%%%%%%%%%%%%%%%%%%%%%%%%%%%%%%%%%%%%%%%%%%%%%%%%%%%%%%%%%%%%%%%%%%%%%%%%%%%%%%%%%%%%%%%%
%CITEVA DEFINITII
\def\om{\omega}
\def\Om{\Omega}
\def\et{\eta}
\def\td{\tilde{\delta}}
\def\m{{\mu}}
\def\n{{\nu}}
\def\k{{\kappa}}
\def\l{{\lambda}}
\def\L{{\Lambda}}
\def\g{{\gamma}}
\def\a{{\alpha}}
\def\e{{\varepsilon}}
\def\b{{\beta}}
\def\G{{\Gamma}}
\def\d{{\delta}}
\def\D{{\Delta}}
\def\t{{\theta}}
\def\s{{\sigma}}
\def\S{{\Sigma}}
\def\z{{\zeta}}
\def\qed{\hfill\Box}
\def\ds{\displaystyle}
\def\mc{\mathcal}
%%%%%%%%%%%%%%%%%%%%%%%%%%%%%%%%%%%%%%%%%%%%%%%%%%%%%%%%%%%%%%%%%%%%%%%%%%%%%%%%%%%%%%%%%%%%%%%%%%%%%%%%%%%%%%%%%%%%%%
\def\1{{\mathbf 1}}
\def\CC{{\mathbb C}}
\def\RR{{\mathbb R}}
\def\QQ{{\mathbb Q}}
\def\ZZ{{\mathbb Z}}
\def\PP{{\mathbb P}}
\def\EE{{\mathbb E}}
\def\NN{{\mathbb N}}
\def\FF{{\mathbb F}}
%\def\SS{{\mathbb S}}
\def\MO{{\mathcal O}}
\def\MF{{\mathcal F}}
\def\MR{{\mathcal R}}
\def\MB{{\mathcal B}}
\def\MM{{\mathcal M}}
\def\MN{{\mathcal N}}
\def\MU{{\mathcal U}}
\def\MP{{\mathcal P}}
\def\MS{{\mathcal S}}
\def\MBS{{\mathbf S}}
\def\MX{{\bm{ \mathscr X}}}
%%%%%%%%%%%%%%%%%%%%%%%%%%%%%%%%%%%%%%%%%%%%%%%%%%%%%%%%%%%%%%%%%%%%%%%%%%%%%%%%%%%%%%%%%%%%%%%%%%%%%%%%%%%%%%%%%%%%%
%Header and Footer
\usepackage{fancyhdr}

\pagestyle{fancy}
\fancyhf{}
\rhead{Universitatea din Bucure\c sti\\ Facultatea de Matematic\u a \c si Informatic\u a}
\lhead{\textit{Curs}: Probabilit\u a\c ti \c si Statistic\u a\\ \textit{Instructori}: A. Am\u arioarei, G. Popovici}
\rfoot{Pagina \thepage}
\lfoot{Grupele: 241, 242, 243, 244}
%%%%%%%%%%%%%%%%%%%%%%%%%%%%%%%%%%%%%%%

%%% Use protect on footnotes to avoid problems with footnotes in titles
\let\rmarkdownfootnote\footnote%
\def\footnote{\protect\rmarkdownfootnote}

%%% Change title format to be more compact
\usepackage{titling}

% Create subtitle command for use in maketitle
\newcommand{\subtitle}[1]{
  \posttitle{
    \begin{center}\large#1\end{center}
    }
}

\setlength{\droptitle}{-2em}
  \title{Tema 1}
  \pretitle{\vspace{\droptitle}\centering\huge}
  \posttitle{\par}
\subtitle{Câmp de probabilitate, operații cu evenimente, probabilitate
conditionată și formula lui Bayes}
  \author{}
  \preauthor{}\postauthor{}
  \date{}
  \predate{}\postdate{}

\begin{document}
\maketitle

%%%%%%%%%%%%%%%%%%%%%%%%
\thispagestyle{fancy}

\subsubsection{\texorpdfstring{Exerci\c tiul
1}{Exerciiul 1}}\label{exerciiul-1}

Fie \(A\), \(B\) \c si \(C\) trei evenimente. Exprima\c ti \(\^ i\)n
func\c tie de \(A\), \(B\), \(C\) \c si de opera\c tiile cu mul\c timi
urm\u atoarele evenimente:

\begin{enumerate}
\def\labelenumi{\alph{enumi})}
\item
  \(A\) singur se realizeaz\u a
\item
  \(A\) \c si \(C\) se realizeaz\u a dar nu \c si \(B\)
\item
  cele trei evenimente se produc
\item
  cel pu\c tin unul dintre cele trei evenimente se produce
\item
  cel pu\c tin dou\u a evenimente din cele trei se produc
\item
  cel mult un eveniment se produce
\item
  niciunul din cele trei evenimente nu se produce
\item
  exact dou\u a evenimente din cele trei se produc
\item
  nu mai mult de dou\u a evenimente nu se realizeaz\u a
\end{enumerate}

\subsubsection{\texorpdfstring{Exerci\c tiul
2}{Exerciiul 2}}\label{exerciiul-2}

Fie \(\Om\) mul\c timea tuturor cuplurilor c\u as\u atorite dintr-un
ora\c s dat. Consider\u am evenimentele urm\u atoare:

\(A\): ``b\u arbatul are mai mult de 40 de ani''

\(B\): ``femeia este mai t\(\^ a\)n\u ar\u a dec\(\^ a\)t b\u arbatul''

\(C\): ``femeia are mai mult de 40 de ani''

Se cere:

\begin{enumerate}
\def\labelenumi{\alph{enumi})}
\item
  Interpreta\c ti in func\c tie de \(A\), \(B\) \c si \(C\) evenimentul
  ``so\c tul are mai mult de 40 de ani dar nu \c si so\c tia sa''.
\item
  Descrie\c ti in limbaj natural evenimentele: \(A\cap B\cap C^{c}\),
  \(A\setminus{(A\cap B)}\) , \(A\cap B^{c}\cap C\), \(A\cup B\)
\item
  Verifica\c ti c\u a \(A\cap C^{c} \subset B\)
\end{enumerate}

\subsubsection{\texorpdfstring{Exerci\c tiul
3}{Exerciiul 3}}\label{exerciiul-3}

Fie \((\Om, \MF, \PP)\) un c\(\^a\)mp de probabilitate. Diferen\c ta
simetric\u a dintre dou\u a evenimente \(A, B\in \MF\) este definit\u a
prin \(A \triangle B=(A\cap B^{c})\cup(A^{c}\cap B)\). Ar\u ata\c ti
c\u a:

\begin{enumerate}
\def\labelenumi{\alph{enumi})}
\item
  func\c tia \(d(A,B)=\PP(A\triangle B)\) este o distan\c t\u a pe
  \(\MF\)
\item
  \(|\PP(A)-\PP(B)|\leq\PP(A\triangle B)\)
\end{enumerate}

\subsubsection{\texorpdfstring{Exerci\c tiul
4}{Exerciiul 4}}\label{exerciiul-4}

Fie \(n\), \(r>1\) numere naturale \c si consider\u am ecua\c tia cu
\(r\) necunoscute: \[
x_1+x_2+\cdots+x_r=n.
\] O solu\c tie a acestei ecua\c tii este un \(r\)-tuplu
\((x_1, \dots, x_r)\) a c\u arui sum\u a a componentelor este \(n\).

\begin{enumerate}
\def\labelenumi{\alph{enumi})}
\item
  C\(\^a\)te solu\c tii are ecua\c tia de mai sus, cu componente
  naturale strict pozitive ?
\item
  C\(\^a\)te solu\c tii are ecua\c tia de mai sus, cu componente
  naturale pozitive ?
\end{enumerate}

\subsubsection{\texorpdfstring{Exerci\c tiul
5}{Exerciiul 5}}\label{exerciiul-5}

Un administrator de spital codific\u a pacien\c tii cu r\u ani prin
\(\^i\)mpu\c scare \(\^i\)n func\c tie de asigurarea (\(D\) dac\u a
pacientul are asigurare \c si \(N\) dac\u a nu are) \c si de starea
acestora (\(b\) pentru bun\u a, \(m\) pentru medie \c si \(s\) pentru
serioas\u a). Consider\u am experien\c ta aleatoare care consist\u a in
codificarea pacien\c tilor.

\begin{enumerate}
\def\labelenumi{\arabic{enumi}.}
\item
  \begin{enumerate}
  \def\labelenumii{\alph{enumii})}
  \tightlist
  \item
    Determina\c ti \(\Om\) mul\c timea spa\c tiului st\u arilor acestui
    experiment
  \item
    Fie \(A\) evenimentul \emph{starea de s\u an\u atate a pacientului
    este serioas\u a}. Descrie\c ti evenimentele elementare care il
    compun pe \(A\). Aceea\c si intrebare pentru evenimentul \(B\)
    \emph{pacientul nu este asigurat} \c si pentru evenimentul
    \(B^c\cup A\).
  \end{enumerate}
\item
  Consider\u am echiprobabilitatea pe \(\Om\). Calcula\c ti
  probabilit\u a\c tile: \(\PP(A)\), \(\PP(B)\) \c si \(\PP(B^c\cup A)\)
\item
  Aceea\c si intrebare pentru probabilitatea \(\PP'\) definit\u a prin
\end{enumerate}

\begin{center}
    \begin{tabular}{c|ccc}
        & $b$ & $m$ & $s$ \\
      \hline
      $O$ & $0,2$ & $0,2$ & $0,1$ \\
      $N$ & $0,1$ & $0,3$ & $0,1$ \\
    \end{tabular}
  \end{center}

\subsubsection{\texorpdfstring{Exerci\c tiul 6 (Jocul de
Poker)}{Exerciiul 6 (Jocul de Poker)}}\label{exerciiul-6-jocul-de-poker}

Consider\u am experimentul in care extragem o man\u a de \(5\)
c\u ar\c ti dintr-un pachet de c\u ar\c ti de \(52\) de c\u ar\c ti.
S\u a se calculeze probabilitatea ca:

\begin{enumerate}
\def\labelenumi{\alph{enumi})}
\item
  S\u a avem careu (patru c\u ar\c ti de acela\c si tip) ?
\item
  S\u a avem full-house (trei c\u ar\c ti de un tip \c si dou\u a de
  altul) ?
\item
  S\u a avem trei c\u ar\c ti de acela\c si tip ?
\item
  S\u a avem dou\u a perechi ?
\item
  S\u a avem o pereche ?
\end{enumerate}

\subsubsection{\texorpdfstring{Exerci\c tiul
7}{Exerciiul 7}}\label{exerciiul-7}

Se dore\c ste verificarea fiabilit\u a\c tii unui test de pentru
depistarea nivelului de alcool al automobili\c stilor. In urma studiilor
statistice pe un num\u ar mare de automobili\c sti, s-a observat c\u a
in general \(0.5\%\) dintre ace\c stia dep\u a\c sesc nivelul de alcool
autorizat. Niciun test nu este fiabil \(100\%\). Probabilitatea ca
testul considerat s\u a fie pozitiv atunci cand doza de alcool
autorizat\u a este dep\u a\c sit\u a precum \c si probabilitatea ca
testul s\u a fie negativ atunci cand doza autorizat\u a nu este
dep\u a\c sit\u a sunt egale cu \(p=0.99\).

\begin{enumerate}
\def\labelenumi{\arabic{enumi}.}
\item
  Care este probabilitatea ca un automobilist care a fost testat pozitiv
  s\u a fi dep\u a\c sit in realitate nivelul de alcool autorizat ?
\item
  C\(\^a\)t devine valoarea parametrului \(p\) pentru ca aceast\u a
  probabilitate s\u a fie de \(95\%\) ?
\item
  Un poli\c tist afirm\u a c\u a testul este mai fiabil samb\u ata seara
  (atunci c\(\^a\)nd tinerii ies din cluburi). \c Stiind c\u a
  propor\c tia de automobili\c sti care au b\u aut prea mult in acest
  context este de \(30\%\), determina\c ti dac\u a poli\c tistul are
  dreptate.
\end{enumerate}

\subsubsection{\texorpdfstring{Exerci\c tiul
8}{Exerciiul 8}}\label{exerciiul-8}

Efectu\u am arunc\u ari succesive a dou\u a zaruri echilibrate \c si
suntem interesa\c ti in g\u asirea probabilit\u a\c tii evenimentului ca
suma \(5\) (a fe\c telor celor dou\u a zaruri) s\u a apar\u a inaintea
sumei \(7\). Pentru aceasta presupunem c\u a arunc\u arile sunt
\emph{independente}.

\begin{enumerate}
\def\labelenumi{\arabic{enumi}.}
\item
  Calcula\c ti pentru inceput probabilitatea evenimentului \(E_n\):
  \emph{in primele \(n-1\) arunc\u ari nu a ap\u arut nici suma \(5\)
  \c si nici suma \(7\) iar in a \(n\)-a aruncare a ap\u arut suma
  \(5\)}. Concluziona\c ti.
\item
  Aceea\c si intrebare, dar inlocuind \(5\) cu \(2\).
\end{enumerate}

\subsubsection[Exerci\c tiul 9* ]{\texorpdfstring{Exerci\c tiul 9*
\footnote{Exerci\c tiile cu * sunt suplimentare \c si nu sunt
  obligatorii}}{Exerciiul 9* }}\label{exerciiul-9}

Fie \((A_1, A_2, \dots, A_n)\), \(n\geq0\) un \c sir de p\u ar\c ti ale
lui \(\Om\) \c si \(\MF\) algebra generat\u a de acesta (cea mai mic\u a
algebr\u a in sensul incluziunii care con\c tine \c mul\c timile
\(\{A_1, A_2, \dots, A_n\}\)). Fie \((c_1, \dots,c_m)\) un \c sir de
numere reale \c si \((B_1, B_2, \dots, B_m)\) un \c sir de elemente din
\(\MF\) (\(m\geq 0\)). Consider\u am inegalitatea

\begin{equation}\label{eq1}
  \displaystyle\sum_{k=1}^{m}c_k\PP(B_k)\geq 0
\end{equation}

unde \(\PP\) este o m\u asur\u a de probabiilitate pe \(\MF\).
Urm\u atoarele propriet\u a\c ti sunt echivalente:

\begin{enumerate}
\def\labelenumi{\alph{enumi})}
\item
  Inegalitatea \((\ref{eq1})\) este adev\u arat\u a pentru toate
  m\u asurile de probabilitate \(\PP\) pe \(\MF\)
\item
  Inegalitatea \((\ref{eq1})\) este adev\u arat\u a pentru toate
  m\u asurile de probabilitate \(\PP\) pe \(\MF\) care verific\u a
  \(\PP(A_i)=0\) sau \(1\), pentru to\c ti \(i\in\{1,2,\dots,n\}\).
\end{enumerate}

\subsubsection{\texorpdfstring{Exerci\c tiul
10*}{Exerciiul 10*}}\label{exerciiul-10}

Fie \(S_{0}^{n}=1\),
\(S_{k}^{n}=\displaystyle\sum_{1\leq i_1<\cdots<i_k\leq n}\PP\left(A_{i_1}\cap\cdots\cap A_{i_k}\right)\)
\c si not\u am cu \(V_{n}^{r}\) (respectiv cu \(W_{n}^{r}\))
probabilitatea ca exact \(r\) (respectiv cel pu\c tin \(r\)) dintre
evenimentele \(A_1, A_2, \dots, A_n\) se realizeaz\u a. Ar\u ata\c ti
c\u a:

\begin{enumerate}
\def\labelenumi{\alph{enumi})}
\item
  \(V_{n}^{r}=\displaystyle\sum_{k=0}^{n-r}(-1)^k\binom{r+k}{k}S_{r+k}^{n}\)
\item
  \(W_{n}^{r}=\displaystyle\sum_{k=0}^{n-r}(-1)^k\binom{r+k-1}{k}S_{r+k}^{n}\)
\end{enumerate}


\end{document}
