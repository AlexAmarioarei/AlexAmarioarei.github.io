\documentclass[]{article}
\usepackage{lmodern}
\usepackage{amssymb,amsmath}
\usepackage{ifxetex,ifluatex}
\usepackage{fixltx2e} % provides \textsubscript
\ifnum 0\ifxetex 1\fi\ifluatex 1\fi=0 % if pdftex
  \usepackage[T1]{fontenc}
  \usepackage[utf8]{inputenc}
\else % if luatex or xelatex
  \ifxetex
    \usepackage{mathspec}
  \else
    \usepackage{fontspec}
  \fi
  \defaultfontfeatures{Ligatures=TeX,Scale=MatchLowercase}
\fi
% use upquote if available, for straight quotes in verbatim environments
\IfFileExists{upquote.sty}{\usepackage{upquote}}{}
% use microtype if available
\IfFileExists{microtype.sty}{%
\usepackage{microtype}
\UseMicrotypeSet[protrusion]{basicmath} % disable protrusion for tt fonts
}{}
\usepackage[margin=1in]{geometry}
\usepackage{hyperref}
\hypersetup{unicode=true,
            pdftitle={Tema 2},
            pdfborder={0 0 0},
            breaklinks=true}
\urlstyle{same}  % don't use monospace font for urls
\usepackage{graphicx,grffile}
\makeatletter
\def\maxwidth{\ifdim\Gin@nat@width>\linewidth\linewidth\else\Gin@nat@width\fi}
\def\maxheight{\ifdim\Gin@nat@height>\textheight\textheight\else\Gin@nat@height\fi}
\makeatother
% Scale images if necessary, so that they will not overflow the page
% margins by default, and it is still possible to overwrite the defaults
% using explicit options in \includegraphics[width, height, ...]{}
\setkeys{Gin}{width=\maxwidth,height=\maxheight,keepaspectratio}
\IfFileExists{parskip.sty}{%
\usepackage{parskip}
}{% else
\setlength{\parindent}{0pt}
\setlength{\parskip}{6pt plus 2pt minus 1pt}
}
\setlength{\emergencystretch}{3em}  % prevent overfull lines
\providecommand{\tightlist}{%
  \setlength{\itemsep}{0pt}\setlength{\parskip}{0pt}}
\setcounter{secnumdepth}{0}
% Redefines (sub)paragraphs to behave more like sections
\ifx\paragraph\undefined\else
\let\oldparagraph\paragraph
\renewcommand{\paragraph}[1]{\oldparagraph{#1}\mbox{}}
\fi
\ifx\subparagraph\undefined\else
\let\oldsubparagraph\subparagraph
\renewcommand{\subparagraph}[1]{\oldsubparagraph{#1}\mbox{}}
\fi
%%%%%%%%%%%%%%%%%%%%%%%%%%%%%%%%%%%%%%%%%%%%%%%%%%%%%%%%%%%%%%%%%%%%%%%%%%%%%%%%%%%%%%%%%%%%%%%%%%%%%%%%%%%%%%%%%%%%%
%%%%%%%%%%%%%%%%%%%%%%%%%%%%%%%%%%%%%%%%%%%%%%%%%%%%%%%%%%%%%%%%%%%%%%%%%%%%%%%%%%%%%%%%%%%%%%%%%%%%%%%%%%%%%%%%%%%%%
%CITEVA DEFINITII
\def\om{\omega}
\def\Om{\Omega}
\def\et{\eta}
\def\td{\tilde{\delta}}
\def\m{{\mu}}
\def\n{{\nu}}
\def\k{{\kappa}}
\def\l{{\lambda}}
\def\L{{\Lambda}}
\def\g{{\gamma}}
\def\a{{\alpha}}
\def\e{{\varepsilon}}
\def\b{{\beta}}
\def\G{{\Gamma}}
\def\d{{\delta}}
\def\D{{\Delta}}
\def\t{{\theta}}
\def\s{{\sigma}}
\def\S{{\Sigma}}
\def\z{{\zeta}}
\def\qed{\hfill\Box}
\def\ds{\displaystyle}
\def\mc{\mathcal}
%%%%%%%%%%%%%%%%%%%%%%%%%%%%%%%%%%%%%%%%%%%%%%%%%%%%%%%%%%%%%%%%%%%%%%%%%%%%%%%%%%%%%%%%%%%%%%%%%%%%%%%%%%%%%%%%%%%%%%
\def\1{{\mathbf 1}}
\def\CC{{\mathbb C}}
\def\VV{{\mathbb V}}
\def\RR{{\mathbb R}}
\def\QQ{{\mathbb Q}}
\def\ZZ{{\mathbb Z}}
\def\PP{{\mathbb P}}
\def\EE{{\mathbb E}}
\def\NN{{\mathbb N}}
\def\FF{{\mathbb F}}
%\def\SS{{\mathbb S}}
\def\MO{{\mathcal O}}
\def\MF{{\mathcal F}}
\def\MR{{\mathcal R}}
\def\MB{{\mathcal B}}
\def\MM{{\mathcal M}}
\def\MN{{\mathcal N}}
\def\MU{{\mathcal U}}
\def\MP{{\mathcal P}}
\def\MS{{\mathcal S}}
\def\MBS{{\mathbf S}}
\def\MX{{\bm{ \mathscr X}}}
%%%%%%%%%%%%%%%%%%%%%%%%%%%%%%%%%%%%%%%%%%%%%%%%%%%%%%%%%%%%%%%%%%%%%%%%%%%%%%%%%%%%%%%%%%%%%%%%%%%%%%%%%%%%%%%%%%%%%
%Header and Footer
\usepackage{fancyhdr}

\pagestyle{fancy}
\fancyhf{}
\rhead{Universitatea din Bucure\c sti\\ Facultatea de Matematic\u a \c si Informatic\u a}
\lhead{\textit{Curs}: Probabilit\u a\c ti \c si Statistic\u a\\ \textit{Instructori}: A. Am\u arioarei, G. Popovici}
\rfoot{Pagina \thepage}
\lfoot{Grupele: 241, 242, 243, 244}
%%%%%%%%%%%%%%%%%%%%%%%%%%%%%%%%%%%%%%%

%%% Use protect on footnotes to avoid problems with footnotes in titles
\let\rmarkdownfootnote\footnote%
\def\footnote{\protect\rmarkdownfootnote}

%%% Change title format to be more compact
\usepackage{titling}

% Create subtitle command for use in maketitle
\newcommand{\subtitle}[1]{
  \posttitle{
    \begin{center}\large#1\end{center}
    }
}

\setlength{\droptitle}{-2em}
  \title{Tema 2}
  \pretitle{\vspace{\droptitle}\centering\huge}
  \posttitle{\par}
  \author{}
  \preauthor{}\postauthor{}
  \date{}
  \predate{}\postdate{}

\begin{document}
\maketitle

%%%%%%%%%%%%%%%%%%%%%%%%
\thispagestyle{fancy}

\subsubsection{\texorpdfstring{Exerci\c tiul
1}{Exerciiul 1}}\label{exerciiul-1}

O urn\u a con\c tine \(r\) bile ro\c sii \c si \(b\) bile albastre. O
bil\u a este extras\u a la intamplare din urn\u a, i se noteaz\u a
culoarea \c si este intoars\u a in urn\u a impreun\u a cu alte \(d\)
bile de aceea\c si culoare. Repet\u am acest proces la nesfar\c sit.
Calcula\c ti:

\begin{enumerate}
\def\labelenumi{\alph{enumi})}
\item
  Probabilitatea ca a doua bil\u a extras\u a s\u a fie albastr\u a.
\item
  Probabilitatea ca prima bil\u a s\u a fie albastr\u a \c stiind c\u a
  a doua bil\u a este albastr\u a.
\item
  Fie \(B_n\) evenimentul ca a \(n\)-a bil\u a extras\u a s\u a fie
  albastr\u a. Ar\u ata\c ti c\u a \(\PP(B_n)=\PP(B_1)\),
  \(\forall\, n\geq1\).
\item
  Probabilitatea ca prima bil\u a este albastr\u a \c stiind c\u a
  urm\u atoarele \(n\) bile extrase sunt albastre. G\u asi\c ti valoarea
  limit\u a a acestei probabilit\u a\c ti.
\end{enumerate}

\subsubsection{\texorpdfstring{Exerci\c tiul
2}{Exerciiul 2}}\label{exerciiul-2}

O companie de asigur\u ari asigur\u a acela\c si num\u ar de
b\u arba\c ti \c si de femei. Intr-un an dat, probabilitatea ca un
b\u arbat s\u a fac\u a accident \c si s\u a aib\u a nevoie de asigurare
este de \(\a\), independent de al\c ti ani. In mod similar,
probabilitatea ca o femeie s\u a fac\u a accident \c si s\u a
beneficieze de asigurare este \(\b\). S\u a presupunem c\u a firma de
asigur\u ri alege la intamplare o persoan\u a.

\begin{enumerate}
\def\labelenumi{\alph{enumi})}
\item
  Care este probabilitatea ca \c soferul ales s\u a aib\u a nevoie de o
  poli\c t\u a de asigurare anul acesta ?
\item
  Care este probabilitatea ca \c soferul ales s\u a aib\u a nevoie de
  asigurare doi ani la rand ?
\item
  Fie \(A_1, A_2\) evenimentele prin care \c soferul ales are nevoie de
  asigurare in primul, respectiv cel de-al doilea an. Ar\u ata\c ti
  c\u a \(\PP(A_2|A_1)\geq\PP(A_1)\).
\item
  G\u asi\c ti probabilitatea ca persoana care are nevoie de asigurare
  s\u a fie o femeie.
\end{enumerate}

\subsubsection{\texorpdfstring{Exerci\c tiul
3}{Exerciiul 3}}\label{exerciiul-3}

Fie \(X\) o variabil\u a aleatoare de densitate: \[
  f(x) = \left\{\begin{array}{ll}
    \frac{2x}{\t^2}, & \mbox{dac\u a $0\leq x\leq \t$}\\
    0, & \mbox{altfel}
  \end{array}\right.
\] unde \(\t\) este un num\u ar pozitiv dat. Determina\c ti func\c tia
de reparti\c tie, media \c si varian\c ta lui \(X\).

\subsubsection{\texorpdfstring{Exerci\c tiul
4}{Exerciiul 4}}\label{exerciiul-4}

Fie \(X\) o variabil\u a aleatoare a c\u arei func\c tie de
reparti\c tie este

\[
  F(x) = \left\{\begin{array}{lll}
    0, & \mbox{dac\u a $x\leq1$}\\
    \ln(x), & \mbox{dac\u a $1<x\leq e$}\\
    1, & \mbox{dac\u a $e<x$}
  \end{array}\right.
\]

Calcula\c ti \(\EE[X]\) \c si \(\VV[X]\).

\subsubsection{\texorpdfstring{Exerci\c tiul
5}{Exerciiul 5}}\label{exerciiul-5}

Fie \(X\) o variabil\u a aleatoare de densitate: \[
  f(x) = \left\{\begin{array}{ll}
    e^{-(x-\t)}, & \mbox{dac\u a $x>\t$}\\
    0, & \mbox{altfel}
  \end{array}\right.
\] unde \(\t\) este un num\u ar real dat.

\begin{enumerate}
\def\labelenumi{\alph{enumi})}
\item
  Determina\c ti func\c tia de reparti\c tie, media, varian\c ta \c si
  mediana acestei variabile
\item
  Fie \(X_1,\cdots,X_n\) variabile aleatoare independente de aceea\c si
  lege ca \c si \(X\) \c si fie \(m_n=\min(X_1,\cdots,X_n)\).
  Determina\c ti func\c tia de reparti\c tie \c si densitatea variabilei
  aleatoare \(m_n\).
\end{enumerate}

\subsubsection{\texorpdfstring{Exerci\c tiul
6}{Exerciiul 6}}\label{exerciiul-6}

Fie \(X\) o variabil\u a aleatoare de densitate: \[
  f(x) = \left\{\begin{array}{ll}
    \frac{1}{\t}x^{\frac{1}{\t}-1}, & \mbox{dac\u a $0\leq x\leq 1$}\\
    0, & \mbox{altfel}
  \end{array}\right.
\] unde \(\t\) este un num\u ar pozitiv dat. Determina\c ti legea
variabilei aleatoare \(Y=-\ln(X)\).


\end{document}
