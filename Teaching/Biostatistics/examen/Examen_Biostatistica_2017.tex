\documentclass[]{article}
\usepackage{lmodern}
\usepackage{amssymb,amsmath}
\usepackage{ifxetex,ifluatex}
\usepackage{fixltx2e} % provides \textsubscript
\ifnum 0\ifxetex 1\fi\ifluatex 1\fi=0 % if pdftex
  \usepackage[T1]{fontenc}
  \usepackage[utf8]{inputenc}
\else % if luatex or xelatex
  \ifxetex
    \usepackage{mathspec}
  \else
    \usepackage{fontspec}
  \fi
  \defaultfontfeatures{Ligatures=TeX,Scale=MatchLowercase}
\fi
% use upquote if available, for straight quotes in verbatim environments
\IfFileExists{upquote.sty}{\usepackage{upquote}}{}
% use microtype if available
\IfFileExists{microtype.sty}{%
\usepackage{microtype}
\UseMicrotypeSet[protrusion]{basicmath} % disable protrusion for tt fonts
}{}
\usepackage[margin=1in]{geometry}
\usepackage{hyperref}
\hypersetup{unicode=true,
            pdftitle={Examen},
            pdfborder={0 0 0},
            breaklinks=true}
\urlstyle{same}  % don't use monospace font for urls
\usepackage{graphicx,grffile}
\makeatletter
\def\maxwidth{\ifdim\Gin@nat@width>\linewidth\linewidth\else\Gin@nat@width\fi}
\def\maxheight{\ifdim\Gin@nat@height>\textheight\textheight\else\Gin@nat@height\fi}
\makeatother
% Scale images if necessary, so that they will not overflow the page
% margins by default, and it is still possible to overwrite the defaults
% using explicit options in \includegraphics[width, height, ...]{}
\setkeys{Gin}{width=\maxwidth,height=\maxheight,keepaspectratio}
\IfFileExists{parskip.sty}{%
\usepackage{parskip}
}{% else
\setlength{\parindent}{0pt}
\setlength{\parskip}{6pt plus 2pt minus 1pt}
}
\setlength{\emergencystretch}{3em}  % prevent overfull lines
\providecommand{\tightlist}{%
  \setlength{\itemsep}{0pt}\setlength{\parskip}{0pt}}
\setcounter{secnumdepth}{0}
% Redefines (sub)paragraphs to behave more like sections
\ifx\paragraph\undefined\else
\let\oldparagraph\paragraph
\renewcommand{\paragraph}[1]{\oldparagraph{#1}\mbox{}}
\fi
\ifx\subparagraph\undefined\else
\let\oldsubparagraph\subparagraph
\renewcommand{\subparagraph}[1]{\oldsubparagraph{#1}\mbox{}}
\fi

%%% Use protect on footnotes to avoid problems with footnotes in titles
\let\rmarkdownfootnote\footnote%
\def\footnote{\protect\rmarkdownfootnote}

%%% Change title format to be more compact
\usepackage{titling}

% Create subtitle command for use in maketitle
\newcommand{\subtitle}[1]{
  \posttitle{
    \begin{center}\large#1\end{center}
    }
}

\setlength{\droptitle}{-2em}
  \title{Examen}
  \pretitle{\vspace{\droptitle}\centering\huge}
  \posttitle{\par}
\subtitle{Timp de lucru 1h30\footnote{Toate documentele și calculatoarele
  electronice de mână sunt autorizate. Computerele personale,
  telefoanele mobile/smartwatch-urile sunt \textbf{strict interzise}.}}
  \author{}
  \preauthor{}\postauthor{}
  \date{}
  \predate{}\postdate{}

%%%%%%%%%%%%%%%%%%%%%%%%%%%%%%%%%%%%%%%%%%%%%%%%%%%%%%%%%%%%%%%%%%%%%%%%%%%%%%%%%%%%%%%%%%%%%%%%%%%%%%%%%%%%%%%%%%%%%
\usepackage{subfigure}
\usepackage{booktabs}
\usepackage{slashbox}
\usepackage{color}
\usepackage{caption}
\usepackage{graphicx}
%%%%%%%%%%%%%%%%%%%%%%%%%%%%%%%%%%%%%%%%%%%%%%%%%%%%%%%%%%%%%%%%%%%%%%%%%%%%%%%%%%%%%%%%%%%%%%%%%%%%%%%%%%%%%%%%%%%%%
%CITEVA DEFINITII
\def\om{\omega}
\def\Om{\Omega}
\def\et{\eta}
\def\td{\tilde{\delta}}
\def\m{{\mu}}
\def\n{{\nu}}
\def\k{{\kappa}}
\def\l{{\lambda}}
\def\L{{\Lambda}}
\def\g{{\gamma}}
\def\a{{\alpha}}
\def\e{{\varepsilon}}
\def\b{{\beta}}
\def\G{{\Gamma}}
\def\d{{\delta}}
\def\D{{\Delta}}
\def\T{{\Theta}}
\def\t{{\theta}}
\def\s{{\sigma}}
\def\S{{\Sigma}}
\def\z{{\zeta}}
\def\qed{\hfill\Box}
\def\ds{\displaystyle}
\def\mc{\mathcal}
%%%%%%%%%%%%%%%%%%%%%%%%%%%%%%%%%%%%%%%%%%%%%%%%%%%%%%%%%%%%%%%%%%%%%%%%%%%%%%%%%%%%%%%%%%%%%%%%%%%%%%%%%%%%%%%%%%%%%%
\def\1{{\mathbf 1}}
\def\CC{{\mathbb C}}
\def\RR{{\mathbb R}}
\def\QQ{{\mathbb Q}}
\def\ZZ{{\mathbb Z}}
\def\PP{{\mathbb P}}
\def\EE{{\mathbb E}}
\def\VV{{\mathbb V}}
\def\NN{{\mathbb N}}
\def\FF{{\mathbb F}}
%\def\SS{{\mathbb S}}
\def\MO{{\mathcal O}}
\def\MA{{\mathcal A}}
\def\MF{{\mathcal F}}
\def\MR{{\mathcal R}}
\def\MB{{\mathcal B}}
\def\MM{{\mathcal M}}
\def\MN{{\mathcal N}}
\def\MU{{\mathcal U}}
\def\MP{{\mathcal P}}
\def\MS{{\mathcal S}}
\def\MBS{{\mathbf S}}
\def\MX{{\bm{ \mathscr X}}}

% independent sign
\newcommand\independent{\protect\mathpalette{\protect\independenT}{\perp}}
\def\independenT#1#2{\mathrel{\rlap{$#1#2$}\mkern2mu{#1#2}}}

%%%%%%%%%%%%%%%%%%%%%%%%%%%%%%%%%%%%%%%%%%%%%%%%%%%%%%%%%%%%%%%%%%%%%%%%%%%%%%%%%%%%%%%%%%%%%%%%%%%%%%%%%%%%%%%%%%%%%
%Header and Footer
\usepackage{fancyhdr}

\pagestyle{fancy}
\fancyhf{}
\rhead{Universitatea din Bucure\c sti\\ Facultatea de Matematic\u a \c si Informatic\u a}
\lhead{\textit{Curs}: Biostatistic\u a 2016-2017\\ \textit{Instructor}: A. Am\u arioarei}
\rfoot{Pagina \thepage}
\lfoot{Grupa: 503}
%%%%%%%%%%%%%%%%%%%%%%%%%%%%%%%%%%%%%%%
\captionsetup[figure]{labelfont={bf},labelformat={default},name={Figura}} 
\captionsetup[table]{labelfont={bf},labelformat={default},name={Tabelul}}

\begin{document}
\maketitle

%%%%%%%%%%%%%%%%%%%%%%%%
\thispagestyle{fancy}

\subsubsection{Exercițiul 1}\label{exercitiul-1}

Să presupunem că observațiile \((x_i,Y_i)\), \(i=1,\ldots,n\) sunt
făcute după modelul \(Y_i = \alpha + \beta x_i + \varepsilon_i\), unde
\(x_1,\ldots,x_n\) sunt constante iar
\(\varepsilon_1,\ldots,\varepsilon_n\sim \mathcal{N}(0,\sigma^2)\) sunt
independente. Modelul este apoi reparametrizat astfel

\[
Y_i = \alpha' + \beta' (x_i-\bar{x}) + \varepsilon_i.
\]

Fie \(\hat{\alpha}\) și \(\hat{\beta}\) estimatorii de verosimilate
maximă a lui \(\alpha\) și \(\beta\), iar \(\hat{\alpha}'\) și
\(\hat{\beta}'\) estimatorii de verosimilitate maximă a lui \(\alpha'\)
și \(\beta'\).

\begin{enumerate}
\def\labelenumi{\alph{enumi})}
\item
  Arătați că \(\hat{\beta}' = \hat{\beta}\).
\item
  Arătați că \(\hat{\alpha}' \neq \hat{\alpha}\) și găsiți repartiția
  lui \(\hat{\alpha}'\).
\item
  Arătați că \(\hat{\alpha}'\) și \(\hat{\beta}'\) sunt necorelați, prin
  urmare sub ipoteza de normalitate sunt independenți.
\end{enumerate}

\subsubsection{Exercițiul 2}\label{exercitiul-2}

Căutăm să exprimăm înălțimea unui arbore \(Y\) în funcție de diametrul
său \(X\) calculat la 1.3 m de sol. Pentru aceasta am măsurat 20 de
cupluri \emph{diametru-înălțime} dintr-o populație normală bivariată cu
parametrii \((\mu_X, \mu_Y, \sigma_X, \sigma_Y, \rho)\) și am obținut
valorile următoare:

\begin{verbatim}
  [,1] [,2] [,3] [,4] [,5] [,6] [,7] [,8] [,9] [,10] [,11] [,12] [,13]
X 38.1 36.1 34.0 37.1 36.8 36.5 36.2 30.4 35.8  36.5  30.8  33.7  35.8
Y 18.6 18.5 18.5 18.5 18.5 18.6 18.5 18.3 18.4  18.5  18.4  18.4  18.5
  [,14] [,15] [,16] [,17] [,18] [,19] [,20]
X  36.9  33.5  33.0  38.2  34.1  31.9  35.9
Y  18.6  18.4  18.4  18.6  18.5  18.3  18.6
\end{verbatim}

\begin{enumerate}
\def\labelenumi{\alph{enumi})}
\item
  Determinați ecuația dreptei de regresie (pentru modelul condiționat)
  \(Y_i = \beta_0 + \beta_1 x_i + \varepsilon_i\).
\item
  Verificați că ipoteza \(H_0:\, \beta_1=0\) este adevărată dacă și
  numai dacă ipoteza \(H_0:\, \rho=0\) este adevărată.
\item
  Arătați că
\end{enumerate}

\[
    \frac{\hat{\beta_1}}{S/\sqrt{S_{xx}}}=\sqrt{n-2}\frac{r}{\sqrt{1-r^2}}
  \]

unde \(r\) este coeficientul de corelație empiric (EVM pentru \(\rho\)).

\begin{enumerate}
\def\labelenumi{\alph{enumi})}
\setcounter{enumi}{3}
\item
  Determinați tabelul ANOVA pentru regresie și testați ipotezele
  \(H_0:\, \beta_j=0\) versus \(H_1:\,\beta_j\neq0\), \(j\in\{0,1\}\).
  Ce credeți despre rezultat ?
\item
  Dați un test pentru ipoteza \(H_0:\, \rho = 0\) care să depindă doar
  de \(r^2\) și de \(n\).
\item
  Să presupunem că am observat un arbore de diametru \(x_0=36\).
  Determinați un interval de predicție pentru înălțimea arborelui.
\end{enumerate}


\end{document}
