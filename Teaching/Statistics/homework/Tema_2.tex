\documentclass[]{article}
\usepackage{lmodern}
\usepackage{amssymb,amsmath}
\usepackage{ifxetex,ifluatex}
\usepackage{fixltx2e} % provides \textsubscript
\ifnum 0\ifxetex 1\fi\ifluatex 1\fi=0 % if pdftex
  \usepackage[T1]{fontenc}
  \usepackage[utf8]{inputenc}
\else % if luatex or xelatex
  \ifxetex
    \usepackage{mathspec}
  \else
    \usepackage{fontspec}
  \fi
  \defaultfontfeatures{Ligatures=TeX,Scale=MatchLowercase}
\fi
% use upquote if available, for straight quotes in verbatim environments
\IfFileExists{upquote.sty}{\usepackage{upquote}}{}
% use microtype if available
\IfFileExists{microtype.sty}{%
\usepackage{microtype}
\UseMicrotypeSet[protrusion]{basicmath} % disable protrusion for tt fonts
}{}
\usepackage[margin=1in]{geometry}
\usepackage{hyperref}
\hypersetup{unicode=true,
            pdftitle={Tema 2},
            pdfborder={0 0 0},
            breaklinks=true}
\urlstyle{same}  % don't use monospace font for urls
\usepackage{graphicx,grffile}
\makeatletter
\def\maxwidth{\ifdim\Gin@nat@width>\linewidth\linewidth\else\Gin@nat@width\fi}
\def\maxheight{\ifdim\Gin@nat@height>\textheight\textheight\else\Gin@nat@height\fi}
\makeatother
% Scale images if necessary, so that they will not overflow the page
% margins by default, and it is still possible to overwrite the defaults
% using explicit options in \includegraphics[width, height, ...]{}
\setkeys{Gin}{width=\maxwidth,height=\maxheight,keepaspectratio}
\IfFileExists{parskip.sty}{%
\usepackage{parskip}
}{% else
\setlength{\parindent}{0pt}
\setlength{\parskip}{6pt plus 2pt minus 1pt}
}
\setlength{\emergencystretch}{3em}  % prevent overfull lines
\providecommand{\tightlist}{%
  \setlength{\itemsep}{0pt}\setlength{\parskip}{0pt}}
\setcounter{secnumdepth}{0}
% Redefines (sub)paragraphs to behave more like sections
\ifx\paragraph\undefined\else
\let\oldparagraph\paragraph
\renewcommand{\paragraph}[1]{\oldparagraph{#1}\mbox{}}
\fi
\ifx\subparagraph\undefined\else
\let\oldsubparagraph\subparagraph
\renewcommand{\subparagraph}[1]{\oldsubparagraph{#1}\mbox{}}
\fi
%%%%%%%%%%%%%%%%%%%%%%%%%%%%%%%%%%%%%%%%%%%%%%%%%%%%%%%%%%%%%%%%%%%%%%%%%%%%%%%%%%%%%%%%%%%%%%%%%%%%%%%%%%%%%%%%%%%%%
%%%%%%%%%%%%%%%%%%%%%%%%%%%%%%%%%%%%%%%%%%%%%%%%%%%%%%%%%%%%%%%%%%%%%%%%%%%%%%%%%%%%%%%%%%%%%%%%%%%%%%%%%%%%%%%%%%%%%
%CITEVA DEFINITII
\def\om{\omega}
\def\Om{\Omega}
\def\et{\eta}
\def\td{\tilde{\delta}}
\def\m{{\mu}}
\def\n{{\nu}}
\def\k{{\kappa}}
\def\l{{\lambda}}
\def\L{{\Lambda}}
\def\g{{\gamma}}
\def\a{{\alpha}}
\def\e{{\varepsilon}}
\def\b{{\beta}}
\def\G{{\Gamma}}
\def\d{{\delta}}
\def\D{{\Delta}}
\def\t{{\theta}}
\def\s{{\sigma}}
\def\S{{\Sigma}}
\def\z{{\zeta}}
\def\qed{\hfill\Box}
\def\ds{\displaystyle}
\def\mc{\mathcal}
%%%%%%%%%%%%%%%%%%%%%%%%%%%%%%%%%%%%%%%%%%%%%%%%%%%%%%%%%%%%%%%%%%%%%%%%%%%%%%%%%%%%%%%%%%%%%%%%%%%%%%%%%%%%%%%%%%%%%%
\def\1{{\mathbf 1}}
\def\CC{{\mathbb C}}
\def\RR{{\mathbb R}}
\def\QQ{{\mathbb Q}}
\def\ZZ{{\mathbb Z}}
\def\PP{{\mathbb P}}
\def\EE{{\mathbb E}}
\def\VV{{\mathbb V}}
\def\NN{{\mathbb N}}
\def\FF{{\mathbb F}}
%\def\SS{{\mathbb S}}
\def\MO{{\mathcal O}}
\def\MF{{\mathcal F}}
\def\MR{{\mathcal R}}
\def\MB{{\mathcal B}}
\def\MM{{\mathcal M}}
\def\MN{{\mathcal N}}
\def\MU{{\mathcal U}}
\def\MP{{\mathcal P}}
\def\MS{{\mathcal S}}
\def\MBS{{\mathbf S}}
\def\MX{{\bm{ \mathscr X}}}
%%%%%%%%%%%%%%%%%%%%%%%%%%%%%%%%%%%%%%%%%%%%%%%%%%%%%%%%%%%%%%%%%%%%%%%%%%%%%%%%%%%%%%%%%%%%%%%%%%%%%%%%%%%%%%%%%%%%%
%Header and Footer
\usepackage{fancyhdr}

\pagestyle{fancy}
\fancyhf{}
\rhead{Universitatea din Bucure\c sti\\ Facultatea de Matematic\u a \c si Informatic\u a}
\lhead{\textit{Curs}: Statistic\u a\\ \textit{Instructori}: A. Am\u arioarei, G. Popovici}
\rfoot{Pagina \thepage}
\lfoot{Grupele: 301, 311, 321}
%%%%%%%%%%%%%%%%%%%%%%%%%%%%%%%%%%%%%%%

%%% Use protect on footnotes to avoid problems with footnotes in titles
\let\rmarkdownfootnote\footnote%
\def\footnote{\protect\rmarkdownfootnote}

%%% Change title format to be more compact
\usepackage{titling}

% Create subtitle command for use in maketitle
\newcommand{\subtitle}[1]{
  \posttitle{
    \begin{center}\large#1\end{center}
    }
}

\setlength{\droptitle}{-2em}
  \title{Tema 2}
  \pretitle{\vspace{\droptitle}\centering\huge}
  \posttitle{\par}
  \author{}
  \preauthor{}\postauthor{}
  \date{}
  \predate{}\postdate{}

\begin{document}
\maketitle

%%%%%%%%%%%%%%%%%%%%%%%%
\thispagestyle{fancy}

\subsubsection{\texorpdfstring{Exerci\c tiul
1}{Exerciiul 1}}\label{exerciiul-1}

Fie \(g:[0,\infty)\to[0, \infty)\) o func\c tie strict cresc\u atoare.
Ar\u ata\c ti c\u a \[
  \PP\left(|X|\geq a\right)\leq\frac{\EE\left[g(|X|)\right]}{g(a)}, \; \mbox{pentru $a>0$.}
\]

\subsubsection{\texorpdfstring{Exerci\c tiul
2}{Exerciiul 2}}\label{exerciiul-2}

Fie \(X\) o variabil\u a aleatoare cu valori in \(\NN\), a\c sa incat
\(p_n=\PP(X=n)>0\) pentru to\c ti \(n\in\NN\).

\begin{enumerate}
\def\labelenumi{\alph{enumi})}
\item
  Ar\u ata\c ti c\u a pentru \(\l>0\) urm\u atoarele afirma\c tii sunt
  echivalente:

  \begin{enumerate}
  \def\labelenumii{\roman{enumii})}
  \item
    \(X\) este o variabil\u a Poisson de parametru \(\l\)
  \item
    Pentru to\c ti \(n\geq1\) avem \(\frac{p_n}{p_{n-1}}=\frac{\l}{n}\)
  \end{enumerate}
\item
  Dac\u a \(X\sim\MP(\l)\) determina\c ti

  \begin{enumerate}
  \def\labelenumii{\roman{enumii})}
  \item
    Valoarea \(k\) pentru care \(\PP(X=k)\) este maxim\u a.
  \item
    Valoarea lui \(\l\) care maximizeaz\u a \(\PP(X=k)\), pentru \(k\)
    fixat.
  \end{enumerate}
\item
  Dac\u a \(X\sim Geom(p)\), \(0<p<1\) calcula\c ti
  \(\EE\left[\frac{1}{X}\right]\).
\end{enumerate}

\subsubsection{\texorpdfstring{Exerci\c tiul
3}{Exerciiul 3}}\label{exerciiul-3}

Ar\u ata\c ti c\u a:

\begin{enumerate}
\def\labelenumi{\alph{enumi})}
\item
  Dac\u a \(X\) este o variabil\u a aleatoare cu valori in \(\NN\)
  atunci \[
    \EE[X] = \displaystyle\sum_{n\geq 1}\PP(X\geq n).
  \]
\item
  Dac\u a \(X\) este o variabil\u a aleatoare cu valori pozitive atunci
  \[
    \EE[X] = \displaystyle\int_{0}^{+\infty}\PP(X\geq x)\, dx.
  \]
\end{enumerate}

\subsubsection{\texorpdfstring{Exerci\c tiul
4}{Exerciiul 4}}\label{exerciiul-4}

\begin{enumerate}
\def\labelenumi{\alph{enumi})}
\tightlist
\item
  Fie \(X\) o variabil\u a repartizat\u a exponen\c tial (de parametru
  \(\a\)). Ar\u ata\c ti c\u a are loc urm\u atoarea relat\c tie
  (proprietatea lipsei de memorie):

  \begin{equation}\label{eq1}
    \PP(X>s+t|X>s) = \PP(X>t) 
  \end{equation}
\item
  Fie \(X\) o variabil\u a aleatoare care verific\u a rela\c tia
  \((\ref{eq1})\). Ar\u ata\c ti c\u a \(X\) este repartizat\u a
  exponen\c tial.
\end{enumerate}

\subsubsection{\texorpdfstring{Exerci\c tiul
5}{Exerciiul 5}}\label{exerciiul-5}

Un proces Bernoulli de parametru \(p\) este un \c sir de variabile
aleatoare independente \((X_n)_{n\geq1}\) cu \(X_n\in\{0,1\}\) \c si
\(\PP(X_n=1)=p\).

\begin{enumerate}
\def\labelenumi{\alph{enumi})}
\item
  Ar\u ata\c ti c\u a v.a. \(S_n=X_1+X_2+\cdots+X_n\) este
  repartizat\u a \(\MB(n,p)\) \c si calcula\u ti media \c si varian\c ta
  acesteia.
\item
  Fie \(L\) cel mai mare num\u ar natural pentru care
  \(X_1=X_2=\cdots=X_L\) \c si \(M\) cel mai mare num\u ar natural
  a\c sa incat \(X_{L+1}=X_{L+2}=\cdots=X_{L+M}\). G\u asi\c ti
  distribu\c tiile v.a. \(L\) \c si \(M\).
\item
  Ar\u ata\c ti c\u a \(\EE[L]\geq\EE[M]\), \(\VV[L]\geq\VV[M]\geq2\)
  \c si calcula\c ti \(Cov[L,M]\).
\item
  Calcula\c ti \(\displaystyle\lim_{k\to\infty}\PP(M=n\,|\,L=k)\).
\end{enumerate}


\end{document}
