\documentclass[]{article}
\usepackage{lmodern}
\usepackage{amssymb,amsmath}
\usepackage{ifxetex,ifluatex}
\usepackage{fixltx2e} % provides \textsubscript
\ifnum 0\ifxetex 1\fi\ifluatex 1\fi=0 % if pdftex
  \usepackage[T1]{fontenc}
  \usepackage[utf8]{inputenc}
\else % if luatex or xelatex
  \ifxetex
    \usepackage{mathspec}
  \else
    \usepackage{fontspec}
  \fi
  \defaultfontfeatures{Ligatures=TeX,Scale=MatchLowercase}
\fi
% use upquote if available, for straight quotes in verbatim environments
\IfFileExists{upquote.sty}{\usepackage{upquote}}{}
% use microtype if available
\IfFileExists{microtype.sty}{%
\usepackage{microtype}
\UseMicrotypeSet[protrusion]{basicmath} % disable protrusion for tt fonts
}{}
\usepackage[margin=1in]{geometry}
\usepackage{hyperref}
\hypersetup{unicode=true,
            pdftitle={Tema 1},
            pdfborder={0 0 0},
            breaklinks=true}
\urlstyle{same}  % don't use monospace font for urls
\usepackage{graphicx,grffile}
\makeatletter
\def\maxwidth{\ifdim\Gin@nat@width>\linewidth\linewidth\else\Gin@nat@width\fi}
\def\maxheight{\ifdim\Gin@nat@height>\textheight\textheight\else\Gin@nat@height\fi}
\makeatother
% Scale images if necessary, so that they will not overflow the page
% margins by default, and it is still possible to overwrite the defaults
% using explicit options in \includegraphics[width, height, ...]{}
\setkeys{Gin}{width=\maxwidth,height=\maxheight,keepaspectratio}
\IfFileExists{parskip.sty}{%
\usepackage{parskip}
}{% else
\setlength{\parindent}{0pt}
\setlength{\parskip}{6pt plus 2pt minus 1pt}
}
\setlength{\emergencystretch}{3em}  % prevent overfull lines
\providecommand{\tightlist}{%
  \setlength{\itemsep}{0pt}\setlength{\parskip}{0pt}}
\setcounter{secnumdepth}{0}
% Redefines (sub)paragraphs to behave more like sections
\ifx\paragraph\undefined\else
\let\oldparagraph\paragraph
\renewcommand{\paragraph}[1]{\oldparagraph{#1}\mbox{}}
\fi
\ifx\subparagraph\undefined\else
\let\oldsubparagraph\subparagraph
\renewcommand{\subparagraph}[1]{\oldsubparagraph{#1}\mbox{}}
\fi
%%%%%%%%%%%%%%%%%%%%%%%%%%%%%%%%%%%%%%%%%%%%%%%%%%%%%%%%%%%%%%%%%%%%%%%%%%%%%%%%%%%%%%%%%%%%%%%%%%%%%%%%%%%%%%%%%%%%%
%%%%%%%%%%%%%%%%%%%%%%%%%%%%%%%%%%%%%%%%%%%%%%%%%%%%%%%%%%%%%%%%%%%%%%%%%%%%%%%%%%%%%%%%%%%%%%%%%%%%%%%%%%%%%%%%%%%%%
%CITEVA DEFINITII
\def\om{\omega}
\def\Om{\Omega}
\def\et{\eta}
\def\td{\tilde{\delta}}
\def\m{{\mu}}
\def\n{{\nu}}
\def\k{{\kappa}}
\def\l{{\lambda}}
\def\L{{\Lambda}}
\def\g{{\gamma}}
\def\a{{\alpha}}
\def\e{{\varepsilon}}
\def\b{{\beta}}
\def\G{{\Gamma}}
\def\d{{\delta}}
\def\D{{\Delta}}
\def\t{{\theta}}
\def\s{{\sigma}}
\def\S{{\Sigma}}
\def\z{{\zeta}}
\def\qed{\hfill\Box}
\def\ds{\displaystyle}
\def\mc{\mathcal}
%%%%%%%%%%%%%%%%%%%%%%%%%%%%%%%%%%%%%%%%%%%%%%%%%%%%%%%%%%%%%%%%%%%%%%%%%%%%%%%%%%%%%%%%%%%%%%%%%%%%%%%%%%%%%%%%%%%%%%
\def\1{{\mathbf 1}}
\def\CC{{\mathbb C}}
\def\RR{{\mathbb R}}
\def\QQ{{\mathbb Q}}
\def\ZZ{{\mathbb Z}}
\def\PP{{\mathbb P}}
\def\EE{{\mathbb E}}
\def\VV{{\mathbb V}}
\def\NN{{\mathbb N}}
\def\FF{{\mathbb F}}
%\def\SS{{\mathbb S}}
\def\MO{{\mathcal O}}
\def\MA{{\mathcal A}}
\def\MF{{\mathcal F}}
\def\MR{{\mathcal R}}
\def\MB{{\mathcal B}}
\def\MM{{\mathcal M}}
\def\MN{{\mathcal N}}
\def\MU{{\mathcal U}}
\def\MP{{\mathcal P}}
\def\MS{{\mathcal S}}
\def\MBS{{\mathbf S}}
\def\MX{{\bm{ \mathscr X}}}
%%%%%%%%%%%%%%%%%%%%%%%%%%%%%%%%%%%%%%%%%%%%%%%%%%%%%%%%%%%%%%%%%%%%%%%%%%%%%%%%%%%%%%%%%%%%%%%%%%%%%%%%%%%%%%%%%%%%%
%Header and Footer
\usepackage{fancyhdr}

\pagestyle{fancy}
\fancyhf{}
\rhead{Universitatea din Bucure\c sti\\ Facultatea de Matematic\u a \c si Informatic\u a}
\lhead{\textit{Curs}: Statistic\u a\\ \textit{Instructori}: A. Am\u arioarei, G. Popovici}
\rfoot{Pagina \thepage}
\lfoot{Grupele: 301, 311, 321}
%%%%%%%%%%%%%%%%%%%%%%%%%%%%%%%%%%%%%%%

%%% Use protect on footnotes to avoid problems with footnotes in titles
\let\rmarkdownfootnote\footnote%
\def\footnote{\protect\rmarkdownfootnote}

%%% Change title format to be more compact
\usepackage{titling}

% Create subtitle command for use in maketitle
\newcommand{\subtitle}[1]{
  \posttitle{
    \begin{center}\large#1\end{center}
    }
}

\setlength{\droptitle}{-2em}
  \title{Tema 1}
  \pretitle{\vspace{\droptitle}\centering\huge}
  \posttitle{\par}
  \author{}
  \preauthor{}\postauthor{}
  \date{}
  \predate{}\postdate{}

\begin{document}
\maketitle

%%%%%%%%%%%%%%%%%%%%%%%%
\thispagestyle{fancy}

\subsubsection{\texorpdfstring{Exerci\c tiul
1}{Exerciiul 1}}\label{exerciiul-1}

Se dore\c ste verificarea fiabilit\u a\c tii unui test de pentru
depistarea nivelului de alcool al automobili\c stilor. In urma studiilor
statistice pe un num\u ar mare de automobili\c sti, s-a observat c\u a
in general \(0.5\%\) dintre ace\c stia dep\u a\c sesc nivelul de alcool
autorizat. Niciun test nu este fiabil \(100\%\). Probabilitatea ca
testul considerat s\u a fie pozitiv atunci cand doza de alcool
autorizat\u a este dep\u a\c sit\u a precum \c si probabilitatea ca
testul s\u a fie negativ atunci cand doza autorizat\u a nu este
dep\u a\c sit\u a sunt egale cu \(p=0.99\).

\begin{enumerate}
\def\labelenumi{\arabic{enumi}.}
\item
  Care este probabilitatea ca un automobilist care a fost testat pozitiv
  s\u a fi dep\u a\c sit in realitate nivelul de alcool autorizat ?
\item
  C\(\^a\)t devine valoarea parametrului \(p\) pentru ca aceast\u a
  probabilitate s\u a fie de \(95\%\) ?
\item
  Un poli\c tist afirm\u a c\u a testul este mai fiabil samb\u ata seara
  (atunci c\(\^a\)nd tinerii ies din cluburi). \c Stiind c\u a
  propor\c tia de automobili\c sti care au b\u aut prea mult in acest
  context este de \(30\%\), determina\c ti dac\u a poli\c tistul are
  dreptate.
\end{enumerate}

\subsubsection{\texorpdfstring{Exerci\c tiul
2}{Exerciiul 2}}\label{exerciiul-2}

Efectu\u am arunc\u ari succesive a dou\u a zaruri echilibrate \c si
suntem interesa\c ti in g\u asirea probabilit\u a\c tii evenimentului ca
suma \(5\) (a fe\c telor celor dou\u a zaruri) s\u a apar\u a inaintea
sumei \(7\). Pentru aceasta presupunem c\u a arunc\u arile sunt
\emph{independente}.

\begin{enumerate}
\def\labelenumi{\arabic{enumi}.}
\item
  Calcula\c ti pentru inceput probabilitatea evenimentului \(E_n\):
  \emph{in primele \(n-1\) arunc\u ari nu a ap\u arut nici suma \(5\)
  \c si nici suma \(7\) iar in a \(n\)-a aruncare a ap\u arut suma
  \(5\)}. Concluziona\c ti.
\item
  Aceea\c si intrebare, dar inlocuind \(5\) cu \(2\).
\end{enumerate}

\subsubsection{\texorpdfstring{Exerci\c tiul
3}{Exerciiul 3}}\label{exerciiul-3}

Un administrator de reprezentan\c t\u a de ma\c sini comand\u a uzinei
Dacia \(N\) ma\c sini, num\u arul aleator \(X\) de ma\c sini pe care il
poate vinde reprezentan\c ta sa intr-un an fiind un num\u ar intreg
intre \(0\) \c si \(n\geq N\), toate avand aceea\c si probabilitate.
Ma\c sinile vandute de administrator ii aduc acestuia un beneficiu de
\(a\) unit\u a\c ti monetare pe ma\c sin\u a iar ma\c sinile nevandute
ii aduc o pierdere de \(b\) unit\u a\c ti. Calcula\c ti valoarea medie a
ca\c stigului \(G\) reprezentan\c tei de ma\c sini \c si deduce\c ti
care este comanda optim\u a.

\subsubsection{\texorpdfstring{Exerci\c tiul
4}{Exerciiul 4}}\label{exerciiul-4}

Fie \(X\) variabila aleatoare (v.a.) care reprezint\u a cifra
ob\c tinut\u a in urma arunc\u arii unui zar (echilibrat) cu \c sase
fe\c te. Determina\c ti legea de probabilitate a v.a. \(Y=X(7-X)\) apoi
calcula\c ti \(\EE[Y]\) \c si \(\VV[Y]\). Not\u am cu
\(Y_1, \dots, Y_n\) valorile observate dup\u a \(n\) lans\u ari
independente. Determina\c ti legea de probabilitate a v.a. \(M_n\)
egal\u a cu valoarea cea mai mare a acestora.

\subsubsection{\texorpdfstring{Exerci\c tiul
5}{Exerciiul 5}}\label{exerciiul-5}

Fie \(Y\) o variabil\u a aleatoare (v.a.) \c si \(m\) mediana ei, i.e.
\(\PP(Y\leq m)\geq\frac{1}{2}\) \c si \(\PP(Y\geq m)\geq\frac{1}{2}\).
Ar\u ata\c ti c\u a pentru orice numere reale \(a\) \c si \(b\) a\c sa
incat \(m\leq a\leq b\) sau \(m\geq a\geq b\) avem \[
\EE|Y-a|\leq\EE|Y-b|.
\]

\subsubsection[Exerci\c tiul 6* ]{\texorpdfstring{Exerci\c tiul 6*
\footnote{Exerci\c tiile cu * sunt suplimentare \c si nu sunt
  obligatorii}}{Exerciiul 6* }}\label{exerciiul-6}

Fie \((A_1, A_2, \dots, A_n)\), \(n\geq0\) un \c sir de p\u ar\c ti ale
lui \(\Om\) \c si \(\MF\) algebra generat\u a de acestea (cea mai
mic\u a algebr\u a in sensul incluziunii care con\c tine mul\c timile
\(\{A_1, A_2, \dots, A_n\}\)). Fie \((c_1, \dots,c_m)\) un \c sir de
numere reale \c si \((B_1, B_2, \dots, B_m)\) un \c sir de elemente din
\(\MF\) (\(m\geq 0\)). Consider\u am inegalitatea

\begin{equation}\label{eq1}
  \displaystyle\sum_{k=1}^{m}c_k\PP(B_k)\geq 0
\end{equation}

unde \(\PP\) este o m\u asur\u a de probabiilitate pe \(\MF\).
Urm\u atoarele propriet\u a\c ti sunt echivalente:

\begin{enumerate}
\def\labelenumi{\alph{enumi})}
\item
  Inegalitatea \((\ref{eq1})\) este adev\u arat\u a pentru toate
  m\u asurile de probabilitate \(\PP\) pe \(\MF\)
\item
  Inegalitatea \((\ref{eq1})\) este adev\u arat\u a pentru toate
  m\u asurile de probabilitate \(\PP\) pe \(\MF\) care verific\u a
  \(\PP(A_i)=0\) sau \(1\), pentru to\c ti \(i\in\{1,2,\dots,n\}\).
\end{enumerate}

\subsubsection{\texorpdfstring{Exerci\c tiul
7*}{Exerciiul 7*}}\label{exerciiul-7}

Fie \(S_{0}^{n}=1\),
\(S_{k}^{n}=\displaystyle\sum_{1\leq i_1<\cdots<i_k\leq n}\PP\left(A_{i_1}\cap\cdots\cap A_{i_k}\right)\)
\c si not\u am cu \(V_{n}^{r}\) (respectiv cu \(W_{n}^{r}\))
probabilitatea ca exact \(r\) (respectiv cel pu\c tin \(r\)) dintre
evenimentele \(A_1, A_2, \dots, A_n\) se realizeaz\u a. Ar\u ata\c ti
c\u a:

\begin{enumerate}
\def\labelenumi{\alph{enumi})}
\item
  \(V_{n}^{r}=\displaystyle\sum_{k=0}^{n-r}(-1)^k\binom{r+k}{k}S_{r+k}^{n}\)
  (Identitatea lui Warning\footnote{Mai este intalnit\u a \c si sub
    numele de de Moivre-Jordan})
\item
  \(W_{n}^{r}=\displaystyle\sum_{k=0}^{n-r}(-1)^k\binom{r+k-1}{k}S_{r+k}^{n}\)
\end{enumerate}


\end{document}
