\documentclass[]{article}
\usepackage{lmodern}
\usepackage{amssymb,amsmath}
\usepackage{ifxetex,ifluatex}
\usepackage{fixltx2e} % provides \textsubscript
\ifnum 0\ifxetex 1\fi\ifluatex 1\fi=0 % if pdftex
  \usepackage[T1]{fontenc}
  \usepackage[utf8]{inputenc}
\else % if luatex or xelatex
  \ifxetex
    \usepackage{mathspec}
  \else
    \usepackage{fontspec}
  \fi
  \defaultfontfeatures{Ligatures=TeX,Scale=MatchLowercase}
\fi
% use upquote if available, for straight quotes in verbatim environments
\IfFileExists{upquote.sty}{\usepackage{upquote}}{}
% use microtype if available
\IfFileExists{microtype.sty}{%
\usepackage{microtype}
\UseMicrotypeSet[protrusion]{basicmath} % disable protrusion for tt fonts
}{}
\usepackage[margin=1in]{geometry}
\usepackage{hyperref}
\hypersetup{unicode=true,
            pdftitle={Tema 2},
            pdfborder={0 0 0},
            breaklinks=true}
\urlstyle{same}  % don't use monospace font for urls
\usepackage{graphicx,grffile}
\makeatletter
\def\maxwidth{\ifdim\Gin@nat@width>\linewidth\linewidth\else\Gin@nat@width\fi}
\def\maxheight{\ifdim\Gin@nat@height>\textheight\textheight\else\Gin@nat@height\fi}
\makeatother
% Scale images if necessary, so that they will not overflow the page
% margins by default, and it is still possible to overwrite the defaults
% using explicit options in \includegraphics[width, height, ...]{}
\setkeys{Gin}{width=\maxwidth,height=\maxheight,keepaspectratio}
\IfFileExists{parskip.sty}{%
\usepackage{parskip}
}{% else
\setlength{\parindent}{0pt}
\setlength{\parskip}{6pt plus 2pt minus 1pt}
}
\setlength{\emergencystretch}{3em}  % prevent overfull lines
\providecommand{\tightlist}{%
  \setlength{\itemsep}{0pt}\setlength{\parskip}{0pt}}
\setcounter{secnumdepth}{0}
% Redefines (sub)paragraphs to behave more like sections
\ifx\paragraph\undefined\else
\let\oldparagraph\paragraph
\renewcommand{\paragraph}[1]{\oldparagraph{#1}\mbox{}}
\fi
\ifx\subparagraph\undefined\else
\let\oldsubparagraph\subparagraph
\renewcommand{\subparagraph}[1]{\oldsubparagraph{#1}\mbox{}}
\fi
%%%%%%%%%%%%%%%%%%%%%%%%%%%%%%%%%%%%%%%%%%%%%%%%%%%%%%%%%%%%%%%%%%%%%%%%%%%%%%%%%%%%%%%%%%%%%%%%%%%%%%%%%%%%%%%%%%%%%
%%%%%%%%%%%%%%%%%%%%%%%%%%%%%%%%%%%%%%%%%%%%%%%%%%%%%%%%%%%%%%%%%%%%%%%%%%%%%%%%%%%%%%%%%%%%%%%%%%%%%%%%%%%%%%%%%%%%%
%CITEVA DEFINITII
\def\om{\omega}
\def\Om{\Omega}
\def\et{\eta}
\def\td{\tilde{\delta}}
\def\m{{\mu}}
\def\n{{\nu}}
\def\k{{\kappa}}
\def\l{{\lambda}}
\def\L{{\Lambda}}
\def\g{{\gamma}}
\def\a{{\alpha}}
\def\e{{\varepsilon}}
\def\b{{\beta}}
\def\G{{\Gamma}}
\def\d{{\delta}}
\def\D{{\Delta}}
\def\t{{\theta}}
\def\s{{\sigma}}
\def\S{{\Sigma}}
\def\z{{\zeta}}
\def\qed{\hfill\Box}
\def\ds{\displaystyle}
\def\mc{\mathcal}
%%%%%%%%%%%%%%%%%%%%%%%%%%%%%%%%%%%%%%%%%%%%%%%%%%%%%%%%%%%%%%%%%%%%%%%%%%%%%%%%%%%%%%%%%%%%%%%%%%%%%%%%%%%%%%%%%%%%%%
\def\1{{\mathbf 1}}
\def\CC{{\mathbb C}}
\def\RR{{\mathbb R}}
\def\QQ{{\mathbb Q}}
\def\ZZ{{\mathbb Z}}
\def\PP{{\mathbb P}}
\def\EE{{\mathbb E}}
\def\VV{{\mathbb V}}
\def\NN{{\mathbb N}}
\def\FF{{\mathbb F}}
%\def\SS{{\mathbb S}}
\def\MO{{\mathcal O}}
\def\MA{{\mathcal A}}
\def\MF{{\mathcal F}}
\def\MR{{\mathcal R}}
\def\MB{{\mathcal B}}
\def\MM{{\mathcal M}}
\def\MN{{\mathcal N}}
\def\MU{{\mathcal U}}
\def\MP{{\mathcal P}}
\def\MS{{\mathcal S}}
\def\MBS{{\mathbf S}}
\def\MX{{\bm{ \mathscr X}}}
%%%%%%%%%%%%%%%%%%%%%%%%%%%%%%%%%%%%%%%%%%%%%%%%%%%%%%%%%%%%%%%%%%%%%%%%%%%%%%%%%%%%%%%%%%%%%%%%%%%%%%%%%%%%%%%%%%%%%
%Header and Footer
\usepackage{fancyhdr}

\pagestyle{fancy}
\fancyhf{}
\rhead{Universitatea din Bucure\c sti\\ Facultatea de Matematic\u a \c si Informatic\u a}
\lhead{\textit{Curs}: Statistic\u a\\ \textit{Instructori}: A. Am\u arioarei, G. Popovici}
\rfoot{Pagina \thepage}
\lfoot{Grupele: 301, 311, 321}
%%%%%%%%%%%%%%%%%%%%%%%%%%%%%%%%%%%%%%%

%%% Use protect on footnotes to avoid problems with footnotes in titles
\let\rmarkdownfootnote\footnote%
\def\footnote{\protect\rmarkdownfootnote}

%%% Change title format to be more compact
\usepackage{titling}

% Create subtitle command for use in maketitle
\newcommand{\subtitle}[1]{
  \posttitle{
    \begin{center}\large#1\end{center}
    }
}

\setlength{\droptitle}{-2em}
  \title{Tema 2}
  \pretitle{\vspace{\droptitle}\centering\huge}
  \posttitle{\par}
\subtitle{Soluții}
  \author{}
  \preauthor{}\postauthor{}
  \date{}
  \predate{}\postdate{}

\begin{document}
\maketitle

%%%%%%%%%%%%%%%%%%%%%%%%
\thispagestyle{fancy}

\subsubsection{\texorpdfstring{Exerci\c tiul
1}{Exerciiul 1}}\label{exerciiul-1}

Observ\u am c\u a pentru \(\om\in\{|X|\geq a\}\) avem
\(\om\in\{g(|X|\geq g(a)\}\) deoarece func\c tia \(g\) este
cresc\u atoare, prin urmare folosind monotonia m\u asurii de
probabilitate \(\PP(|X|\geq a)\leq\PP(g(|X|\geq g(a))\). Dac\u a
consider\u am \(A=\{g(|X|\geq g(a)\}\), atunci \(g(|X|)\geq g(a)\1_{A}\)
ceea ce implic\u a \(\EE[g(|X|)]\geq g(a)\EE[\1_A]=g(a)\PP(A)\). Ultima
rela\c tie arat\u a c\u a \(\PP(A)\leq\frac{\EE[g(|X|)]}{g(a)}\),
deoarece pentru \(a>0\) avem \(g(a)>0\) (\(g\) este strict
cresc\u atoare \c si pozitiv\u a). In final ob\c tinem \[
\PP(|X|\geq a)\leq\PP(g(|X|\geq g(a))=\PP(A)\leq\frac{\EE[g(|X|)]}{g(a)}.
\]

\subsubsection{\texorpdfstring{Exerci\c tiul
2}{Exerciiul 2}}\label{exerciiul-2}

\begin{enumerate}
\def\labelenumi{\alph{enumi})}
\item
  Dac\u a i) este adev\u arat\u a atunci \(p_n=e^{-\l}\frac{\l^n}{n!}\)
  de unde ob\c tinem imediat ii). Reciproc, presupunem ii)
  adev\u arat\u a \c si avem c\u a
  \(\frac{p_n}{p_0}=\frac{p_1}{p_0}\frac{p_2}{p_1}\cdots\frac{p_n}{p_{n-1}}=\frac{\l^n}{n!}\)
  de unde \(p_n=p_0\frac{\l^n}{n!}\). Cum \(\sum_{k=0}^{\infty}p_k=1\)
  ob\c tinem c\u a \(p_0=e^{-\l}\) \c si putem s\u a concluzion\u am.
\item
  \begin{enumerate}
  \def\labelenumii{\roman{enumii})}
  \item
    \c Stim c\u a \(\PP(X=j)=\frac{\l^j}{j!}e^{-\l}\) \c si vrem s\u a
    evalu\u am raportul \(\frac{\PP(X=j)}{\PP(X=j-1)}\): \[
    \frac{\PP(X=j)}{\PP(X=j-1)}=\frac{\frac{\l^j}{j!}e^{-\l}}{\frac{\l^{j-1}}{(j-1)!}e^{-\l}}=\frac{\l}{j}.
    \] Putem observa c\u a \[
    \begin{array}{ll}
     \PP(X=j)\geq\PP(X=j-1), & \mbox{dac\u a $\l\geq j$}\\
      \PP(X=j)<\PP(X=j-1), & \mbox{dac\u a $\l<j$}.\end{array}
      \] ceea ce arat\u a c\u a \(j=[\l]\) este punctul maxim \c si
    \(\PP(X=[\l])=\frac{\l^{[\l]}}{[\l]!}e^{-\l}\) este valoarea
    maxim\u a.
  \item
    Dup\u a cum am v\u azut la punctul precedent avem
    \(\frac{\PP(X=j)}{\PP(X=j-1)}=\frac{\l}{j}\). Dac\u a \(j>0\) este
    fixat atunci putem observa c\u a maximum este atins pentru \(\l=j\).
  \end{enumerate}
\item
  Dac\u a \(X\sim Geom(p)\) atunci
  \(X=\sum_{k\geq1}p(1-p)^{k-1}\e_{k}\), unde \(\e_k\) este m\u asura
  Dirac in \(k\). Avem \[
    \EE\left[\frac{1}{X}\right] = \EE\left[\frac{1}{1+X-1}\right] = \EE\left[\int_{0}^{1}t^{X-1}\,dt\right] = \int_{0}^{1}\EE[t^{X-1}]\,dt
  \] iar \[
    \EE[t^X] = \sum_{k=1}^{\infty}t^{k}\PP(X=k) = \sum_{k=1}^{\infty}t^{k}(1-p)^k\frac{p}{1-p}=\frac{pt}{1-(1-p)t}
  \] ceea ce ne conduce la \[
    \EE\left[\frac{1}{X}\right] = \int_{0}^{1}\frac{1}{t}\frac{pt}{1-(1-p)t}\, dt = -\frac{p}{1-p}\ln(p).
  \] Dac\u a alegem s\u a definim variabila geometric\u a prin
  \(X=\sum_{k\geq0}p(1-p)^{k}\e_{k}\) atunci
  \(\EE\left[\frac{1}{X}\right]=+\infty\).
\end{enumerate}

\subsubsection{\texorpdfstring{Exerci\c tiul
3}{Exerciiul 3}}\label{exerciiul-3}

\begin{enumerate}
\def\labelenumi{\alph{enumi})}
\item
  Observ\u am c\u a:
  \[\PP(X>n)=\PP(X=n+1)+\PP(X=n+2)+\dots=\sum_{k=n+1}^{\infty}{\PP(X=k)}\]
  \c si

  \begin{align*}
  \sum_{n=0}^{\infty}{\sum_{k=n+1}^{\infty}{\PP(X=k)}}&=(\PP(X=1)+\PP(X=2)+\dots)+(\PP(X=2)+\PP(X=3)+\dots)\\
    &+(\PP(X=3)+\PP(X=4)+\dots)+\dots\\
    &=1\cdot\PP(X=1)+2\cdot\PP(X=2)+3\cdot\PP(X=3)+\dots\\
    &=\sum_{k=1}^{\infty}{k\PP(X=k)}=\sum_{k=0}^{\infty}{k\PP(X=k)}=\EE[X].
  \end{align*}

  O alt\u a idee ar fi s\u a lu\u am
  \(X=\sum_{i=0}^{\infty}\1_{\{X>i\}}\) \c si s\u a invers\u am \(\sum\)
  cu \(\EE\) (de ce putem ?).
\item
  Avem \[
    \displaystyle\int_{0}^{+\infty}\PP(X\geq x)\, dx = \displaystyle\int_{0}^{+\infty}\EE[\1_{\{X\geq x\}}]\, dx \stackrel{\text{Tonelli-Fubini}}{=} \EE\left[\int_{0}^{\infty}\1_{\{X\geq x\}}\, dx\right] = \EE\left[\int_{0}^{X}\, dx\right] = \EE[X].
  \]
\end{enumerate}

\subsubsection{\texorpdfstring{Exerci\c tiul
4}{Exerciiul 4}}\label{exerciiul-4}

\begin{enumerate}
\def\labelenumi{\alph{enumi})}
\tightlist
\item
  Dac\u a \(X\sim Exp(\a)\) atunci densitatea sa este
  \(f(x) = \a e^{-\a x}\1_{(0,\infty)}(x)\) \c si are func\c tia de
  reparti\c tie \(F(x) = (1-e^{-\a x})\1_{(0,\infty)}(x)\). Observ\u am
  c\u a \(\PP(X>t) = 1-\PP(X\leq t) = 1-F(t) = e^{-\a t}\), deci

  \begin{align*}
  \PP(X>t+s|X>s) &= \frac{\PP(X>t+s,X>s)}{\PP(X>s)} = \frac{\PP(X>t+s)}{\PP(X>s)} = \frac{e^{-\a(t+s)}}{e^{-\a st}} = e^{-\a t} = \PP(X>t).
  \end{align*}
\end{enumerate}

\underline{Interpretare:} S\u a presupunem c\u a \(X\) reprezint\u a
durata de via\c t\u a a unei componente electrice. \c Stiind c\u a \(X\)
a func\c tionat deja un timp \(s\) (\(X>s\)), probabilitatea ca \(X\)
s\u a func\c tioneze un timp \(t\) suplimentar (\(X>t+s\)) este egal\u a
cu probabilitatea ca \(X\) s\u a func\c tioneze un timp \(t\). Faptul
c\u a a func\c tionat deja un timp \(s\) nu influen\c teaz\u a
probabilitatea s\u a func\c tioneze un timp \(t\) in plus.

\begin{enumerate}
\def\labelenumi{\alph{enumi})}
\setcounter{enumi}{1}
\tightlist
\item
  Din rela\c tia \[
  \PP(X>s+t|X>s)=\frac{\PP(X>s+t,X>s)}{\PP(X>s)}=\frac{\PP(X>s+t)}{\PP(X>s)}=\PP(X>t)
  \] ob\c tinem \(\PP(X>s+t)=\PP(X>s)\PP(X>t)\) de unde notand cu
  \(h(t)=\PP(X>t)\) avem \(h(s+t)=h(s)h(t)\) pentru \(s>0,\ t>0\).
\end{enumerate}

Pentru a verifica c\u a v.a. \(X\) este distribuit\u a exponen\c tial
trebuie s\u a rezolv\u am ecua\c tia func\c tional\u a Cauchy.
Observ\u am pentru inceput c\u a dac\u a \(s=t\) atunci \(h(2s)=h^2(s)\)
\c si prin induc\c tie avem \(h(ks)=h^k(s)\) pentru \(k\in\NN\). Luand
\(s=\frac{1}{2}\) avem \(h(1)=h^2(\frac{1}{2})\) de unde
\(h(\frac{1}{2})=h^{\frac{1}{2}}(1)\) \c si pentru \(s=\frac{1}{k}\)
rezult\u a c\u a \(h(\frac{1}{k})=h^{\frac{1}{k}}(1)\) (prin aceea\c si
argumentare). Combinand rezultatele avem
\(h(\frac{m}{n})=h(m\frac{1}{n})=h^m(\frac{1}{n})=h^{\frac{m}{n}}(1)\).
Prin urmare \(h(q)=h^q(1)\) pentru orice \(q\in\QQ_+\). Dac\u a
\(r\in \RR_+-\QQ_+\) exist\u a un \c sir \((q_n)_n\subset\QQ_+\) a\c sa
incat \(q_n\downarrow r\) \c si folosind continuitatea la dreapta avem
\(h(q_n)\downarrow h(r)\) deci \(h(r)=a^r\), unde \(a=h(1)\). In final
am g\u asit c\u a \(h(t)=e^{-t\log{\frac{1}{h(1)}}}\).

\subsubsection{\texorpdfstring{Exerci\c tiul
5}{Exerciiul 5}}\label{exerciiul-5}

\begin{enumerate}
\def\labelenumi{\alph{enumi})}
\item
  Este defini\c tia binomialei. Avem \(\EE[S_n]=np\) \c si
  \(\VV[S_n] = np(1-p)\).
\item
  Avem c\u a
  \(\{L=n\}=\{X_1=\cdots=X_n=1,X_{n+1}=0\}\cup\{X_1=\cdots=X_n=0,X_{n+1}=1\}\)
  de unde \(\PP(L=n)=p^nq+pq^n\), \(n\geq1\), \(q=1-p\). Rezult\u a
  c\u a

  \begin{align*}
  \EE[L] &= \sum_{n\geq1} n\PP(L=n) = \sum_{n\geq1} n(p^nq+pq^n) = 2+\frac{(p-q)^2}{pq}\\
  \VV[L] &= \sum_{n\geq1} n^2\PP(L=n) - \EE[L]^2 = 2+\frac{(1+pq)(p-q)^2}{p^2q^2}
    \end{align*}

  Pentru a g\u asi legea lui \(M\) s\u a ne uit\u am la cuplul \((L,M)\)
  \c si s\u a observ\u am c\u a evenimentul \(\{L=n, M=m\}\) este dat de
  \(\{X_1=\cdots=X_n=1,X_{n+1}=\cdots=X_{n+m}=0,X_{n+m+1}=1\}\cup\{X_1=\cdots=X_n=0,X_{n+1}=\cdots=X_{n+m}=1,X_{n+m+1}=0\}\)
  de unde

  \begin{align*}
  \PP(L=n,M=m) &= \PP(X_1=\cdots=X_n=1,X_{n+1}=\cdots=X_{n+m}=0,X_{n+m+1}=1) +\\           &\; \PP(X_1=\cdots=X_n=0,X_{n+1}=\cdots=X_{n+m}=1,X_{n+m+1}=0)= p^nq^mp+p^mq^nq
    \end{align*}

  \c si prin urmare legea lui \(M\) este

  \begin{align*}
  \PP(M=m) &= \sum_{n\geq1}\PP(L=n,M=m)= q^{m-1}p^2+p^{m-1}q^2.
    \end{align*}

  Ob\c tinem asfel c\u a \(\EE[M]=2\) (independent de \(p\)) \c si c\u a
  \[
    \VV[M] = 2+\frac{2(p-q)^2}{pq}.
  \]
\item
  Este evident c\u a \(\EE[L]=2+\frac{(p-q)^2}{pq} \geq 2=\EE[M]\) \c si
  c\u a
  \(\VV[L]=2+\frac{(1+pq)(p-q)^2}{p^2q^2}\geq 2+\frac{2(p-q)^2}{pq}=\VV[M]\geq 2\).
  Se poate calcula u\c sor c\u a \[
  \EE[LM] = \sum_{n,m\geq1}nm\PP(L=n,M=m) = \frac{1}{pq}
    \] de unde rezult\u a c\u a \(Cov[L,M] = -\frac{(p-q)^2}{pq}\).
\item
  Observ\u am c\u a \[
  \displaystyle\lim_{k\to\infty}\PP(M=n\,|\,L=k)=\lim_{k\to\infty}\frac{\PP(L=k,M=n)}{\PP(L=k)}=\lim_{k\to\infty}\frac{p^{k+1}q^n+q^{k+1}p^{n}}{p^kq+q^kp}
    \] \c si studiind comportamentul raportului \(\frac{p}{q}\) (dac\u a
  este \(>1\) sau nu dup\u a cum e \(p\)) deducem c\u a \[
    \displaystyle\lim_{k\to\infty}\PP(M=n\,|\,L=k)=\left\{\begin{array}{lll}
      p^{n-1}q, & \mbox{dac\u a $p<\frac{1}{2}$}\\
      q^{n-1}p, & \mbox{dac\u a $p>\frac{1}{2}$}\\
      2^{-n}, & \mbox{dac\u a $p=\frac{1}{2}$}\\
    \end{array}\right.
  \]
\end{enumerate}


\end{document}
