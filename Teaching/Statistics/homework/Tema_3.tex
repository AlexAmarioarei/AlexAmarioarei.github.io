\documentclass[]{article}
\usepackage{lmodern}
\usepackage{amssymb,amsmath}
\usepackage{ifxetex,ifluatex}
\usepackage{fixltx2e} % provides \textsubscript
\ifnum 0\ifxetex 1\fi\ifluatex 1\fi=0 % if pdftex
  \usepackage[T1]{fontenc}
  \usepackage[utf8]{inputenc}
\else % if luatex or xelatex
  \ifxetex
    \usepackage{mathspec}
  \else
    \usepackage{fontspec}
  \fi
  \defaultfontfeatures{Ligatures=TeX,Scale=MatchLowercase}
\fi
% use upquote if available, for straight quotes in verbatim environments
\IfFileExists{upquote.sty}{\usepackage{upquote}}{}
% use microtype if available
\IfFileExists{microtype.sty}{%
\usepackage{microtype}
\UseMicrotypeSet[protrusion]{basicmath} % disable protrusion for tt fonts
}{}
\usepackage[margin=1in]{geometry}
\usepackage{hyperref}
\hypersetup{unicode=true,
            pdftitle={Tema 3},
            pdfborder={0 0 0},
            breaklinks=true}
\urlstyle{same}  % don't use monospace font for urls
\usepackage{graphicx,grffile}
\makeatletter
\def\maxwidth{\ifdim\Gin@nat@width>\linewidth\linewidth\else\Gin@nat@width\fi}
\def\maxheight{\ifdim\Gin@nat@height>\textheight\textheight\else\Gin@nat@height\fi}
\makeatother
% Scale images if necessary, so that they will not overflow the page
% margins by default, and it is still possible to overwrite the defaults
% using explicit options in \includegraphics[width, height, ...]{}
\setkeys{Gin}{width=\maxwidth,height=\maxheight,keepaspectratio}
\IfFileExists{parskip.sty}{%
\usepackage{parskip}
}{% else
\setlength{\parindent}{0pt}
\setlength{\parskip}{6pt plus 2pt minus 1pt}
}
\setlength{\emergencystretch}{3em}  % prevent overfull lines
\providecommand{\tightlist}{%
  \setlength{\itemsep}{0pt}\setlength{\parskip}{0pt}}
\setcounter{secnumdepth}{0}
% Redefines (sub)paragraphs to behave more like sections
\ifx\paragraph\undefined\else
\let\oldparagraph\paragraph
\renewcommand{\paragraph}[1]{\oldparagraph{#1}\mbox{}}
\fi
\ifx\subparagraph\undefined\else
\let\oldsubparagraph\subparagraph
\renewcommand{\subparagraph}[1]{\oldsubparagraph{#1}\mbox{}}
\fi

%%% Use protect on footnotes to avoid problems with footnotes in titles
\let\rmarkdownfootnote\footnote%
\def\footnote{\protect\rmarkdownfootnote}

%%% Change title format to be more compact
\usepackage{titling}

% Create subtitle command for use in maketitle
\newcommand{\subtitle}[1]{
  \posttitle{
    \begin{center}\large#1\end{center}
    }
}

\setlength{\droptitle}{-2em}
  \title{Tema 3}
  \pretitle{\vspace{\droptitle}\centering\huge}
  \posttitle{\par}
\subtitle{(predare pana pe 04.11.16)}
  \author{}
  \preauthor{}\postauthor{}
  \date{}
  \predate{}\postdate{}

%%%%%%%%%%%%%%%%%%%%%%%%%%%%%%%%%%%%%%%%%%%%%%%%%%%%%%%%%%%%%%%%%%%%%%%%%%%%%%%%%%%%%%%%%%%%%%%%%%%%%%%%%%%%%%%%%%%%%
\usepackage{subfigure}
\usepackage{booktabs}
\usepackage{slashbox}

%%%%%%%%%%%%%%%%%%%%%%%%%%%%%%%%%%%%%%%%%%%%%%%%%%%%%%%%%%%%%%%%%%%%%%%%%%%%%%%%%%%%%%%%%%%%%%%%%%%%%%%%%%%%%%%%%%%%%
%CITEVA DEFINITII
\def\om{\omega}
\def\Om{\Omega}
\def\et{\eta}
\def\td{\tilde{\delta}}
\def\m{{\mu}}
\def\n{{\nu}}
\def\k{{\kappa}}
\def\l{{\lambda}}
\def\L{{\Lambda}}
\def\g{{\gamma}}
\def\a{{\alpha}}
\def\e{{\varepsilon}}
\def\b{{\beta}}
\def\G{{\Gamma}}
\def\d{{\delta}}
\def\D{{\Delta}}
\def\t{{\theta}}
\def\s{{\sigma}}
\def\S{{\Sigma}}
\def\z{{\zeta}}
\def\qed{\hfill\Box}
\def\ds{\displaystyle}
\def\mc{\mathcal}
%%%%%%%%%%%%%%%%%%%%%%%%%%%%%%%%%%%%%%%%%%%%%%%%%%%%%%%%%%%%%%%%%%%%%%%%%%%%%%%%%%%%%%%%%%%%%%%%%%%%%%%%%%%%%%%%%%%%%%
\def\1{{\mathbf 1}}
\def\CC{{\mathbb C}}
\def\RR{{\mathbb R}}
\def\QQ{{\mathbb Q}}
\def\ZZ{{\mathbb Z}}
\def\PP{{\mathbb P}}
\def\EE{{\mathbb E}}
\def\VV{{\mathbb V}}
\def\NN{{\mathbb N}}
\def\FF{{\mathbb F}}
%\def\SS{{\mathbb S}}
\def\MO{{\mathcal O}}
\def\MA{{\mathcal A}}
\def\MF{{\mathcal F}}
\def\MR{{\mathcal R}}
\def\MB{{\mathcal B}}
\def\MM{{\mathcal M}}
\def\MN{{\mathcal N}}
\def\MU{{\mathcal U}}
\def\MP{{\mathcal P}}
\def\MS{{\mathcal S}}
\def\MBS{{\mathbf S}}
\def\MX{{\bm{ \mathscr X}}}
%%%%%%%%%%%%%%%%%%%%%%%%%%%%%%%%%%%%%%%%%%%%%%%%%%%%%%%%%%%%%%%%%%%%%%%%%%%%%%%%%%%%%%%%%%%%%%%%%%%%%%%%%%%%%%%%%%%%%
%Header and Footer
\usepackage{fancyhdr}

\pagestyle{fancy}
\fancyhf{}
\rhead{Universitatea din Bucure\c sti\\ Facultatea de Matematic\u a \c si Informatic\u a}
\lhead{\textit{Curs}: Statistic\u a\\ \textit{Instructor}: A. Am\u arioarei}
\rfoot{Pagina \thepage}
\lfoot{Grupele: 301, 311, 321}
%%%%%%%%%%%%%%%%%%%%%%%%%%%%%%%%%%%%%%%

\begin{document}
\maketitle

%%%%%%%%%%%%%%%%%%%%%%%%
\thispagestyle{fancy}

\subsubsection{\texorpdfstring{Exerci\c tiul
1}{Exerciiul 1}}\label{exerciiul-1}

O urn\u a con\c tine \(r\) bile ro\c sii \c si \(b\) bile albastre. O
bil\u a este extras\u a la intamplare din urn\u a, i se noteaz\u a
culoarea \c si este intoars\u a in urn\u a impreun\u a cu alte \(d\)
bile de aceea\c si culoare. Repet\u am acest proces la nesfar\c sit.
Calcula\c ti:

\begin{enumerate}
\def\labelenumi{\alph{enumi})}
\item
  Probabilitatea ca a doua bil\u a extras\u a s\u a fie albastr\u a.
\item
  Probabilitatea ca prima bil\u a s\u a fie albastr\u a \c stiind c\u a
  a doua bil\u a este albastr\u a.
\item
  Fie \(B_n\) evenimentul ca a \(n\)-a bil\u a extras\u a s\u a fie
  albastr\u a. Ar\u ata\c ti c\u a \(\PP(B_n)=\PP(B_1)\),
  \(\forall\, n\geq1\).
\item
  Probabilitatea ca prima bil\u a este albastr\u a \c stiind c\u a
  urm\u atoarele \(n\) bile extrase sunt albastre. G\u asi\c ti valoarea
  limit\u a a acestei probabilit\u a\c ti.
\end{enumerate}

\subsubsection{\texorpdfstring{Exerci\c tiul
2}{Exerciiul 2}}\label{exerciiul-2}

\c Stim c\u a intr-un lot de \(5\) tranzistori avem \(2\) care sunt
defec\c ti. Tranzistorii sunt testa\c ti, unul cate unul, pan\u a cand
cei doi tranzistori au fost identifica\c ti. Fie \(N_1\) num\u arul de
teste pentru identificarea primului tranzistor defect \c si \(N_2\)
num\u arul de teste suplimentare pentru identificarea celui de-al doilea
tranzistor defect. Scrie\c ti un tablou in care s\u a descrie\c ti legea
cuplului \((N_1,N_2)\). Calcula\c ti \(\EE[N_1]\) \c si \(\EE[N_2]\).

\subsubsection{\texorpdfstring{Exerci\c tiul
3}{Exerciiul 3}}\label{exerciiul-3}

Fie \((X_1,X_2)\) vectorul aleator distribuit uniform pe discul \(D(R)\)
centrat in origine \c si de raz\u a \(R\). Densitatea vectorului
\((X_1,X_2)\) este dat\u a de \[
  f(x_1,x_2) = c\1_{D(R)}(x_1,x_2)
\] unde \(c\) este o constant\u a pozitiv\u a.

\begin{enumerate}
\def\labelenumi{\arabic{enumi}.}
\item
  Determina\c ti constanta \(c\).
\item
  Determina\c ti legile marginale ale lui \(X_1\) \c si \(X_2\).
\item
  Fie \(L\) distan\c ta de la punctul \((X_1,X_2)\) la origine.
  G\u asi\c ti func\c tia de reparti\c tie a lui \(L\), legea lui \(L\)
  \c si media sa.
\end{enumerate}


\end{document}
