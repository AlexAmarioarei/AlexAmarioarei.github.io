\usepackage{booktabs}
\usepackage{longtable}
\usepackage{framed,color}
\definecolor{shadecolor}{RGB}{248, 248, 248}
\definecolor{shadecolor1}{RGB}{216,225,235}
\definecolor{framecolor}{RGB}{16,111,124}%108,123,13

%\definecolor{shadecolor}{RGB}{226, 255, 241}
% \definecolor{shadecolor1}{RGB}{217,225,199}
% \definecolor{framecolor}{RGB}{60,179,113}

%%%%%%%%%%%%%%%%%%%%%%
\ifxetex
  \usepackage{letltxmacro}
  \setlength{\XeTeXLinkMargin}{1pt}
  \LetLtxMacro\SavedIncludeGraphics\includegraphics
  \def\includegraphics#1#{% #1 catches optional stuff (star/opt. arg.)
    \IncludeGraphicsAux{#1}%
  }%
  \newcommand*{\IncludeGraphicsAux}[2]{%
    \XeTeXLinkBox{%
      \SavedIncludeGraphics#1{#2}%
    }%
  }%
\fi

\newenvironment{frshaded*}{%
  \def\FrameCommand{\fboxrule=\FrameRule\fboxsep=\FrameSep \fcolorbox{framecolor}{shadecolor1}}%
  \MakeFramed {\advance\hsize-\width \FrameRestore}}%
{\endMakeFramed}

\newenvironment{rmdblock}[1]
  {\begin{frshaded*}
  \begin{itemize}
  \renewcommand{\labelitemi}{
    \raisebox{-.7\height}[0pt][0pt]{
      {\setkeys{Gin}{width=2em,keepaspectratio}\includegraphics{images/icons/#1}}
    }
  }
  \item
  }
  {
  \end{itemize}
  \end{frshaded*}
  }
  
%%%%%%%%%%%%%%%
% definitions.
% -------------------
\usepackage{marginnote}
% \renewcommand*{\marginnotevadjust}{40pt}
% \renewcommand{\marginnotevadjust}{0pt}
% \renewcommand{\marginfont}{\noindent\rule{0pt}{0.7\baselineskip}\tiny}

\newtheorem{proposition}{Proposition}[section]
\newtheorem{lemma}[proposition]{Lemma}
\newtheorem{corollary}[proposition]{Corollary}
\newtheorem{theorem}[proposition]{Theorem}

\newcounter{exo}[section]
\newcommand{\enonce}[2]{\refstepcounter{proposition}\hypertarget{exo:#1}{}\label{exo:#1}{\scriptsize\;\textbf{Ex.}~\ref{exo:#1}}}

\reversemarginpar
\setlength{\marginparwidth}{1.2cm}
% 
% \newcommand{\enonce}[2]{\refstepcounter{proposition}\hypertarget{exo:#1}{}\label{exo:#1}{\noindent\color{black}\normalsize\bf Exercice \ref{exo:#1}}\ \  #2\vspace{1mm}\hrule\vspace{1mm} \color{black}\normalsize}


%%%%%%%%%%%%%%%

% \newenvironment{rmdcaution}
%   {\begin{rmdblock}{caution}}
%   {\end{rmdblock}}

% \newenvironment{rmdinsight}
%   {\begin{rmdblock}{insight}}
%   {\end{rmdblock}}

\newenvironment{rmdexercise}
  {\begin{rmdblock}{exercise}}
  {\end{rmdblock}}

% \newenvironment{rmdexercise_tex}
%   {\begin{rmdblock}{exercise}}
%   {\end{rmdblock}}
  
% \newenvironment{rmdtip}
%   {\begin{rmdblock}{tip}}
%   {\end{rmdblock}}


%%%%%%%%%%%%%%%%%%%%%%%%%%%%%%%%%%%%%%%%%%%%%%%%%%%%%%%%%%%%%%%%%%%%%%%%%%%%%%%%%%%%%%%%%%%%%%%%%%%%%%%%%%%%%%%%%%%%%
%%%%%%%%%%% For insight block %%%%%%%%%%%%%%%%%%%%%%%%%%
\definecolor{shadecolor_insight}{RGB}{223,240,216}
\definecolor{framecolor_insight}{RGB}{136,193,137}

%\definecolor{shadecolor_insight}{RGB}{217,225,199}
%\definecolor{framecolor_insight}{RGB}{60,179,113}

\newenvironment{frshaded_insight*}{%
  \def\FrameCommand{\fboxrule=\FrameRule\fboxsep=\FrameSep \fcolorbox{framecolor_insight}{shadecolor_insight}}%
  \MakeFramed {\advance\hsize-\width \FrameRestore}}%
{\endMakeFramed}

\newenvironment{rmdblock_insight}[1]
  {\begin{frshaded_insight*}
  \begin{itemize}
  \renewcommand{\labelitemi}{
    \raisebox{-.7\height}[0pt][0pt]{
      {\setkeys{Gin}{width=2em,keepaspectratio}\includegraphics{images/icons/#1}}
    }
  }
  \item
  }
  {
  \end{itemize}
  \end{frshaded_insight*}
  }

\newenvironment{rmdinsight}
  {\begin{rmdblock_insight}{insight}}
  {\end{rmdblock_insight}}

%%%%%%%%%%% For caution block %%%%%%%%%%%%%%%%%%%%%%%%%%
\definecolor{shadecolor_caution}{RGB}{250,250,250}
\definecolor{framecolor_caution}{RGB}{242,129,67}%193,75,34

\newenvironment{frshaded_caution*}{%
  \def\FrameCommand{\fboxrule=\FrameRule\fboxsep=\FrameSep \fcolorbox{framecolor_caution}{shadecolor_caution}}%
  \MakeFramed {\advance\hsize-\width \FrameRestore}}%
{\endMakeFramed}

\newenvironment{rmdblock_caution}[1]
  {\begin{frshaded_caution*}
  \begin{itemize}
  \renewcommand{\labelitemi}{
    \raisebox{-.7\height}[0pt][0pt]{
      {\setkeys{Gin}{width=2em,keepaspectratio}\includegraphics{images/icons/#1}}
    }
  }
  \item
  }
  {
  \end{itemize}
  \end{frshaded_caution*}
  }
  
\newenvironment{rmdcaution}
  {\begin{rmdblock_caution}{caution}}
  {\end{rmdblock_caution}}

%%%%%%%%%%% For tip block %%%%%%%%%%%%%%%%%%%%%%%%%%
\definecolor{shadecolor_tip}{RGB}{250,250,250}
\definecolor{framecolor_tip}{RGB}{33,153,195}

\newenvironment{frshaded_tip*}{%
  \def\FrameCommand{\fboxrule=\FrameRule\fboxsep=\FrameSep \fcolorbox{framecolor_tip}{shadecolor_tip}}%
  \MakeFramed {\advance\hsize-\width \FrameRestore}}%
{\endMakeFramed}

\newenvironment{rmdblock_tip}[1]
  {\begin{frshaded_tip*}
  \begin{itemize}
  \renewcommand{\labelitemi}{
    \raisebox{-.7\height}[0pt][0pt]{
      {\setkeys{Gin}{width=2em,keepaspectratio}\includegraphics{images/icons/#1}}
    }
  }
  \item
  }
  {
  \end{itemize}
  \end{frshaded_tip*}
  }
  
\newenvironment{rmdtip}
  {\begin{rmdblock_tip}{tip}}
  {\end{rmdblock_tip}}

%%%%%%%%%%%%%%%%%%%%%%%%%%%%%%%%%%%%%%%%%%%%%%%%%%%%%%%%%%%%%%%%%%%%%%%%%%%%%%%%%%%%%%%%%%%%%%%%%%%%%%%%%%%%%%%%%%%%%
\usepackage{subfigure}
\usepackage{booktabs}
\usepackage{slashbox}
\usepackage{color}
%%%%%%%%%%%%%%%%%%%%%%%%%%%%%%%%%%%%%%%%%
\definecolor{linkcol}{rgb}{0,0,0.4}
\definecolor{citecol}{rgb}{0.5,0,0}

% Change this to change the informations included in the pdf file
% \usepackage[pagebackref]{hyperref}
% \usepackage[verbose]{backref}
\usepackage[hyperpageref]{backref}
% \backrefsetup{verbose=false}
% \PassOptionsToPackage{pagebackref}{hyperref}
% See hyperref documentation for information on those parameters

\hypersetup
{
bookmarksopen=true,
pdftitle="Curs Probabilitati si Statistica",
pdfauthor="Alexandru Amarioarei",
pdfsubject="Curs Probabilitati si Statistica", %subject of the document
pdfmenubar=true, %menubar shown
pdfhighlight=/O, %effect of clicking on a link
colorlinks=true, %couleurs sur les liens hypertextes
pdfpagemode=None, %aucun mode de page
pdfpagelayout=SinglePage, %ouverture en simple page
pdffitwindow=true, %pages ouvertes entierement dans toute la fenetre
linkcolor=linkcol, %couleur des liens hypertextes internes
citecolor=citecol, %couleur des liens pour les citations
urlcolor=linkcol %couleur des liens pour les url
}


% set the back references
\renewcommand*{\backref}[1]{}
\renewcommand*{\backreftwosep}{ și~} % inserted between entries 
                              % in a list of two entries, 
                              % default is " and~".
\renewcommand*{\backreflastsep}{ și~} % inserted between the last 
                               % two entries of a list with more
                               % than two entries, default is ", and~".
\renewcommand*{\backrefalt}[4]{%
    \ifcase #1 (Necitat.)%
    \or        (Citat la pagina~#2.)%
    \else      (Citat la paginile~#2.)%
    \fi}

%%%%%%%%%%%%%%%%%%%%%%%%%%%%%%%%%%%%%%%%%%%%%%%%%%%%%%%%%%%%%%%%%%%%%%%%%%%%%%%%%%%%%%%%%%%%%%%%%%%%%%%%%%%%%%%%%%%%%
%CITEVA DEFINITII
\def\om{\omega}
\def\Om{\Omega}
\def\et{\eta}
\def\td{\tilde{\delta}}
\def\m{{\mu}}
\def\n{{\nu}}
\def\k{{\kappa}}
\def\l{{\lambda}}
\def\L{{\Lambda}}
\def\g{{\gamma}}
\def\a{{\alpha}}
\def\e{{\varepsilon}}
\def\b{{\beta}}
\def\G{{\Gamma}}
\def\d{{\delta}}
\def\D{{\Delta}}
\def\t{{\theta}}
\def\s{{\sigma}}
\def\S{{\Sigma}}
\def\z{{\zeta}}
\def\qed{\hfill\Box}
\def\ds{\displaystyle}
\def\mc{\mathcal}
%%%%%%%%%%%%%%%%%%%%%%%%%%%%%%%%%%%%%%%%%%%%%%%%%%%%%%%%%%%%%%%%%%%%%%%%%%%%%%%%%%%%%%%%%%%%%%%%%%%%%%%%%%%%%%%%%%%%%%
\def\1{{\mathbf 1}}
\def\CC{{\mathbb C}}
\def\VV{{\mathbb V}}
\def\RR{{\mathbb R}}
\def\QQ{{\mathbb Q}}
\def\ZZ{{\mathbb Z}}
\def\PP{{\mathbb P}}
\def\EE{{\mathbb E}}
\def\NN{{\mathbb N}}
\def\FF{{\mathbb F}}
%\def\SS{{\mathbb S}}
\def\MA{{\mathcal A}}
\def\MO{{\mathcal O}}
\def\MF{{\mathcal F}}
\def\ME{{\mathcal E}}
\def\MR{{\mathcal R}}
\def\MB{{\mathcal B}}
\def\MM{{\mathcal M}}
\def\MN{{\mathcal N}}
\def\MU{{\mathcal U}}
\def\MP{{\mathcal P}}
\def\MS{{\mathcal S}}
\def\MBS{{\mathbf S}}
\def\MX{{\bm{ \mathscr X}}}

% independent sign
\newcommand\independent{\protect\mathpalette{\protect\independenT}{\perp}}
\def\independenT#1#2{\mathrel{\rlap{$#1#2$}\mkern2mu{#1#2}}}

\renewcommand\tablename{Tab.}
\renewcommand{\figurename}{Fig.}
\renewcommand\refname{Referințe}

%%%%%%%%%%%%%%%%%%%%%%%%%%%%%%%%%%%%%%%%%%%%%%%%%%%%%%%%%%%%%%%%%%%%%%%%%%%%%%%%%%%%%%%%%%%%%%%%%%%%%%%%%%%%%%%%%%%%%
%Header and Footer
\usepackage{fancyhdr}

\pagestyle{fancy}
\fancyhf{}
\rhead{Universitatea din Bucure\c sti\\ Facultatea de Matematic\u a \c si Informatic\u a}
% \lhead{\textit{Curs}: Tehnici alternative în predarea matematicii (2018)\\ \textit{Instructori}: A. Am\u arioarei, M. Patriche}
\lhead{\textit{Curs}: Probabilități și Statistică (2018-2019)\\ \textit{Instructor}: A. Am\u arioarei}
\rfoot{Pagina \thepage}
\lfoot{Grupele: 241, 242, 243, 244}
%%%%%%%%%%%%%%%%%%%%%%%%%%%%%%%%%%%%%%%
%%%%%%%%%%%%%%%%%%%%%%%%%%%%%%%%%%%%%%%
\usepackage{pifont}% http://ctan.org/pkg/pifont
\newcommand{\cmark}{\ding{51}}%
\newcommand{\xmark}{\ding{55}}%

